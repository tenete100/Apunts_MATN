\documentclass{report}
% Configuración
%----------------
%   Importaciones
%----------------
\usepackage{xparse,xcolor}
\usepackage{sectsty}
\usepackage{amsmath,amssymb,latexsym,cancel,amsthm,mathtools} %amsfonts
\usepackage{tikz,pgfplots}
\pgfplotsset{compat=1.18, width=10cm}
\usepackage{multicol}
\usepackage{tikz-cd}
\usepackage{enumitem}

\usepackage[colorlinks=true]{hyperref}
\usepackage[most]{tcolorbox}
\usepackage{blindtext}
\usepackage{framed}
\usepackage{titletoc}
\usepackage{etoolbox}

\usepackage[explicit]{titlesec}
\usepackage{anyfontsize}

\usepackage[a4paper]{geometry}
\usepackage{graphicx, wrapfig, subcaption, setspace, booktabs}
\usepackage[T1]{fontenc}
\usepackage[catalan]{babel}
\usepackage[scaled]{helvet}     % Fuente del documento


\renewcommand{\familydefault}{\rmdefault} % per tornar a canviar la font posar aquí \sfdefault
\usepackage[utf8]{inputenc}
\usepackage{url, lipsum}
\usepackage{tabularx}
\usepackage{cancel}

% \setlength{\parindent}{0cm}
% \setlength{\parskip}{5pt}

%-----------------
%   Colores
%-----------------
\definecolor{mytheorembg}{HTML}{F2F2F9}
\definecolor{mylenmabg}{HTML}{FFFAF8}
\definecolor{mylenmafr}{HTML}{983b0f}
\definecolor{mypropbg}{HTML}{f2fbfc}
\definecolor{mypropfr}{HTML}{191971}
\definecolor{myp}{RGB}{197, 92, 212}
\definecolor{primary}{HTML}{207ba5}    % Color principal
\definecolor{migris}{RGB}{17, 17, 17}
\definecolor{grisfondo}{RGB}{249, 249, 249}
\definecolor{MyGrey}{HTML}{5B5B5B}


%----------------
%   Colores para
%   Las urls
%----------------
\hypersetup{
    colorlinks=true,
    linkcolor=black,
    filecolor=magenta,
    urlcolor=blue,
}

%----------------
%   Cajas
%----------------


% \newcommand{\comb}[2]{\begin{pmatrix}
%      #1\\
%      #2
% \end{pmatrix}}

\newcommand\fancybox[3]{%
    \tcbset{
        mybox/.style={
                enhanced,
                boxsep=0mm,
                opacityfill=0,
                overlay={
                        \coordinate (X) at ([xshift=-1mm, yshift=-1.5mm]frame.north west);
                        \node[align=right, text=#1, text width=2.5cm, anchor=north east] at (X) {\bf#2};
                        \draw[line width=0.5mm, color=#1] (frame.north west) -- (frame.south west);
                    }
            }
    }
    \begin{tcolorbox}[mybox]
        #3
    \end{tcolorbox}
}

\tcbuselibrary{theorems,skins,hooks}
\NewDocumentCommand\caja{m O{\Large #1} O{grisfondo} O{primary} O{number within=chapter}}
{
    \newtcbtheorem[#5]{#1}{\large #2}
    {%
        enhanced
        ,breakable
        ,colback = #3
        ,frame hidden
        ,boxrule = 0sp
        ,borderline west = {2pt}{0pt}{#4}
        ,sharp corners
        ,detach title
        ,before upper = \tcbtitle\par\smallskip
        ,coltitle = #4
        ,fonttitle = \bfseries%
        % ,description font = \mdseries
        ,separator sign none
        ,segmentation style={solid, #4}
    }
    {th}
}

\caja{Corolari}[Corol·lari][myp!10][myp!85!black]
\caja{Lema}[Lema][mylenmabg][mylenmafr]
\caja{Propo}[Proposició][mypropbg][mypropfr]
\caja{defi}[Definició][primary!12][primary]
\caja{Teorema}[Teorema][primary!12][primary]
\caja{Nota}[Nota][white][migris][no counter]
\caja{Propietats}[Propietats][white][migris][no counter]
\caja{preg}[Pregunta][white][migris][no counter]
\caja{Claim}[Claim:][white][migris][no counter]

%---------------
%   Comandos
%---------------
\newcommand{\teorema}[2]{\begin{Teorema}{#1}{}#2\end{Teorema}}
\newcommand{\pregunta}[2]{\begin{preg}{#1}{}#2\end{preg}}
\newcommand{\coro}[2]{\begin{Corolari}{#1}{}#2\end{Corolari}}
\newcommand{\lema}[2]{\begin{Lema}{#1}{}#2\end{Lema}}
\newcommand{\Prop}[2]{\begin{Propo}{#1}{}#2\end{Propo}}
\newcommand{\propietats}[2]{\begin{Propietats}{#1}{}#2\end{Propietats}}
\newcommand{\nota}[2]{\begin{Nota}{#1}{}{\em\color{MyGrey}#2}\end{Nota}}
\newcommand{\dfe}[2]{\begin{defi}{#1}{}#2\end{defi}}
\newcommand{\pf}[1]{\begin{proof}[\color{primary}\textbf{Demostració.}] #1 \end{proof}\vspace{7mm}}
\newcommand{\claim}[1]{\begin{Claim}{}{} #1 \end{Claim}}

\theoremstyle{definition}
\newtheorem*{exem}{\color{primary}Exemple}
\newcommand{\exemple}[1]{\begin{exem}#1\end{exem}}

\theoremstyle{definition}
\newtheorem*{solu}{\color{primary}Solució}
\newcommand{\sol}[1]{\begin{solu}#1\end{solu}}

\theoremstyle{definition}
\newtheorem*{obser}{\color{primary}Observació}
\newcommand{\obs}[1]{\begin{obser}#1\end{obser}}

%---------------
%   Listas
%---------------

% \newcommand{\cnumero}[2]{
%     \tikz[baseline=(myanchor.base)]
%     \node[minimum size=0.2cm,circle,
%         inner sep=1pt,draw, #2,thick,fill=#2](myanchor)
%     {\color{white}\bfseries\fontsize{8}{8}#1};}

\newcommand*{\itembolasazules}[1]{\protect\cnumero{#1}{primary}}
    
\newcommand{\listo}[1]{
    \begin{enumerate}[label=\itembolasazules{\arabic*}]
        #1
    \end{enumerate}
}

\newcommand{\listu}[1]{
    \begin{itemize}[label=$\color{primary} \bullet$]
        #1
    \end{itemize}
}

%-------------------------
% Tabla de Contenidos
%-------------------------

\patchcmd{\tableofcontents}{\contentsname}{\contentsname}{}{}

\renewenvironment{leftbar}
{\def\FrameCommand{\hspace{6em}%
        {\color{primary}\vrule width 2pt depth 6pt}\hspace{1em}}%
    \MakeFramed{\parshape 1 0cm \dimexpr\textwidth-6em\relax\FrameRestore}\vskip2pt%
}
{\endMakeFramed}

\titlecontents{chapter}[0em]
{\vspace*{2\baselineskip}}
{\parbox{4.5em}{%
        \hfill\Huge\bfseries\color{primary}\thecontentslabel}
    \vspace*{-2.3\baselineskip}\leftbar\textbf{\color{primary}\small\chaptername~\thecontentslabel}\\
}{}{\endleftbar}

\titlecontents{section}[8.4em]
{\contentslabel{3em}}{}{} 
{\hspace{0.5em}\nobreak\itshape\color{primary}\contentspage}

\titlecontents{subsection}[8.4em]
{\contentslabel{3em}}{}{} 
{\hspace{0.5em}\nobreak\itshape\color{primary}\contentspage}

%-----------------------------
%   Formato de los capitulos
%-----------------------------

%==================
% Capitulos
%==================
\newtcolorbox{titlecolorbox}[1]{ %the box around chapter
    coltext=white,
    colframe=primary,
    colback=primary,
    boxrule=0pt,
    arc=0pt,
    notitle,
    width=4.8em,
    height=2.4ex,
    before=\hfill
}


\makeatletter
\let\old@rule\@rule
\def\@rule[#1]#2#3{\textcolor{primary}{\old@rule[#1]{#2}{#3}}}
\makeatother

\titleformat{\chapter}[display]
{\Huge}
{}
{0pt}
{\begin{titlecolorbox}{}
        {\large\MakeUppercase{\bf\chaptername}}
    \end{titlecolorbox}
    \vspace*{-4.19ex}\noindent\rule{\textwidth}{0.4pt}
    \parbox[b]{\dimexpr\textwidth-4.8em\relax}{\raggedright\MakeUppercase{#1}}{\hfill\fontsize{70}{60}\selectfont{\color{primary}\thechapter}}
}
[]

\titleformat{name=\chapter,numberless}[display]
{\Huge\normalfont}
{}
{0pt}
{
    \vspace*{-4.19ex}\noindent\rule{\textwidth}{0.4pt}
    \parbox[b]{\dimexpr\textwidth-4.8em\relax}{\raggedright\MakeUppercase{#1}}
}
[]

%==============
% Secciones
%==============

\titleformat{\section}[hang]{\Large\normalfont\bfseries}
{\rlap{\color{primary}\rule[-6pt]{\textwidth}{0.4pt}}\colorbox{primary}{%
        \raisebox{0pt}[13pt][3pt]{ \makebox[60pt]{% height, width
                \fontfamily{cmr}\selectfont\color{white}{\thesection}}
        }}}%
{15pt}%
{ \color{primary}#1
    %
}
\titlespacing*{\section}{0pt}{3mm}{5mm}

%================
% Sub secciones
%================
\subsectionfont{\Large\color{primary}}

%---------------------
% Portada
%---------------------
\usetikzlibrary{ shapes.geometric }
\usetikzlibrary{calc}
\newcommand{\portada}[3]{
    \begin{tikzpicture}[remember picture,overlay]
        %%%%%%%%%%%%%%%%%%%% Background %%%%%%%%%%%%%%%%%%%%%%%%
        %\fill[primary] (current page.south west) rectangle (current page.north east);


        \foreach \i in {2.5,...,22}
            {
                \node[rounded corners,black!40,draw,regular polygon,regular polygon sides=6, minimum size=\i cm,ultra thick] at ($(current page.west)+(2.5,-5)$) {} ;
            }

        %%%%%%%%%%%%%%%%%%%% Background Polygon %%%%%%%%%%%%%%%%%%%% 
        \foreach \i in {0.5,...,22}
            {
                \node[rounded corners,black!40,draw,regular polygon,regular polygon sides=6, minimum size=\i cm,ultra thick] at ($(current page.north west)+(2.5,0)$) {} ;
            }

        \foreach \i in {0.5,...,22}
            {
                \node[rounded corners,black!20,draw,regular polygon,regular polygon sides=6, minimum size=\i cm,ultra thick] at ($(current page.north east)+(0,-9.5)$) {} ;
            }


        \foreach \i in {21,...,6}
            {
                \node[black!85,rounded corners,draw,regular polygon,regular polygon sides=6, minimum size=\i cm,ultra thick] at ($(current page.south east)+(-0.2,-0.45)$) {} ;
            }


        %%%%%%%%%%%%%%%%%%%% Title of the Report %%%%%%%%%%%%%%%%%%%% 
        \node[left,black!70,minimum width=0.625*\paperwidth,minimum height=3cm, rounded corners] at ($(current page.north east)+(0,-9.5)$)
        {
            {\fontsize{25}{30} \selectfont \bfseries #1} 
        };

        %%%%%%%%%%%%%%%%%%%% Subtitle %%%%%%%%%%%%%%%%%%%% 
        \node[left,black!60,minimum width=0.625*\paperwidth,minimum height=2cm, rounded corners] at ($(current page.north east)+(0,-11)$)
        {
            {\huge \textit{#2}}
        };

        %%%%%%%%%%%%%%%%%%%% Author Name %%%%%%%%%%%%%%%%%%%% 
        \node[left,black!60,minimum width=0.625*\paperwidth,minimum height=2cm, rounded corners] at ($(current page.north east)+(0,-13)$)
        {
            {\Large \textsc{#3}}
        };

        %%%%%%%%%%%%%%%%%%%% Year %%%%%%%%%%%%%%%%%%%% 
        \node[rounded corners,fill=black!70,text =primary!5,regular polygon,regular polygon sides=6, minimum size=2.5 cm,inner sep=0,ultra thick] at ($(current page.west)+(2.5,-5)$) {\LARGE \bfseries \the\year{}};

    \end{tikzpicture}
}

% declaracions de dreceres
\providecommand{\ol}{\overline}
\providecommand{\ul}{\underline}
\providecommand{\wt}{\widetilde}
\providecommand{\wh}{\widehat}
\providecommand{\eps}{\varepsilon}
\providecommand{\half}{\frac{1}{2}}
\providecommand{\inv}{^{-1}}
\newcommand{\dang}{\measuredangle} %% Directed angle
\providecommand{\CC}{\mathbb C}
\providecommand{\s}{\sigma}
\providecommand{\FF}{\mathbb F}
\providecommand{\KK}{\mathbb K}
\providecommand{\NN}{{\mathbb Z_{\geq1}}}
\providecommand{\QQ}{\mathbb Q}
\providecommand{\RR}{\mathbb R}
\providecommand{\ZZ}{\mathbb Z}
\providecommand{\PP}{\mathbb P}
\providecommand{\bigO}{\mathcal O}
\newcommand\smallO{
  \mathchoice
    {{\scriptstyle\mathcal{O}}}% \displaystyle
    {{\scriptstyle\mathcal{O}}}% \textstyle
    {{\scriptscriptstyle\mathcal{O}}}% \scriptstyle
    {\scalebox{.6}{$\scriptscriptstyle\mathcal{O}$}}%\scriptscriptstyle
  }
\providecommand{\dd}{\mathrm{d}}

\providecommand{\ts}{\textsuperscript}
\providecommand{\dg}{^\circ}
\providecommand{\ii}{\item}
\DeclareMathOperator*{\lcm}{lcm}
\DeclareMathOperator*{\argmin}{arg min}
\DeclareMathOperator*{\argmax}{arg max}
\DeclareMathOperator{\Ima}{Im}
\DeclareMathOperator{\Hom}{Hom}
\newcommand{\mcal}[1]{\mathcal{#1}}

\begin{document}
\renewcommand{\rmdefault}{cmr}

\pagestyle{empty}
\portada{Mètodes analítics en teoria de nombres}{}{Bernat Esteve i Luis M. Villabón}
\newpage

\tableofcontents

\newpage
\chapter{Definicions}
\section{Sèries de Dirichlet}
\dfe{Sèrie de Dirichlet}{Una sèrie de Dirichlet és una sèrie de la forma
\[
\sum_{n\in\NN}\frac{f(n)}{n^s}
\]
On $f\colon\NN\rightarrow\CC$ és una funció aritmètica, i $s\in\CC$.
}

\dfe{Convergència de productoris}{
Sigui $\{z_n\}_n\subset\CC$ una successió de complexos, aleshores el productori $\prod z_n$ convergeix si i només si existeix el límit $\lim_{n\rightarrow\infty}\prod_{i=1}^nz_n$, i aquest és no nul.
}

\dfe{Convergència absoluta de productoris}{
Donada una seqüència $z_n$ amb $\Re(z_n)>0$, aleshores el productori $\prod z_r$ es diu que convergeix absolutament si $\sum \log(z_r)$ convergeix absolutament.
}
\section{Funcions L}
\dfe{Caràcter d'un grup}{
Sigui $G$ un grup finit i abelià, aleshores un caràcter de $G$ serà un morfisme $\psi\colon G\rightarrow \CC^*$ (on $\CC^*$ és el grup multiplicatiu de $\CC\setminus\{0\}$)}

\dfe{El grup de caràcters}{Denotem per $\widehat{G}$ al grup $\widehat{G}=\Hom(G,\CC^*)=\{\text{Caràcters de G}\}$, on la operació és:
\[
\psi,\phi\colon G\rightarrow\CC^*\qquad\text{aleshores }(\psi{\;\cdot_{\widehat{G}} } \;\phi)(g)\mapsto\psi(g)\cdot_{\small{\mathbb{C}}^*}\phi(g)
\]
I anomenarem a $\widehat{G}$ el grup de caràcters de $G$.
}

\dfe{Caràcter mòdul $m$}{
Un caràcter mòdul $m$ és un caràcter de $(\mathbb{Z}/m\mathbb{Z})^\times$.}

\dfe{Caràcter principal mòdul $m$}{El caràcter principal mòdul $m$ és $\chi_0(a) \colon \ZZ^+ \to \CC$ definit per 
\[
    \chi_0(a) = \begin{cases}
        1 & \gcd(a, m) = 1 \\
        0 & \text{altrament}
    \end{cases}
\]}

\dfe{Funció $L$ associada a un caràcter de Dirichlet}{La funció $L$ associada al caràcter de Dirichlet mòdul $m$ $\chi$ és la sèrie de Dirichlet \[
L(\chi, s) = \sum_{n\geq 1} \frac{\chi(n)}{n^s}.
\].}

\dfe{Funció $m$-èsima de Dirichlet}{Sigui $m\in\NN$, aleshores la funció $m$-èsima de Dirichlet es defineix com
\[
\zeta_m(s)\vcentcolon=\prod_\chi L(\chi,s)
\]
On el producte recorre tots els caràcters de Dirichlet mòdul $m$.
}

\dfe{Caràcters reals i complexos.}{Diem que $\chi\colon\NN\to \CC$ és un caràcter de Dirichlet real si $\Ima(\chi)\subset\RR$. És a dir $\Ima(\chi)\subset\{-1,0,1\}$. I direm que és complex altrament.}

\dfe{Caràcter conjugat}{Direm $\overline{X}$ al caràcter de Dirichlet conjugat
\[
\overline{X}\colon\NN\to\CC\qquad\text{que envia } a\in\ZZ\text{ a }\overline{\chi}(a)=\overline{\chi(a)}
\]
I no és massa difícil de veure que aquesta funció és un caràcter de Dirichlet del mateix mòdul.}
\section{Continuació meromorfa i equació funcional de $\zeta(s)$}

\dfe{$\theta$ de Jacobi}{
La funció $\theta$ de Jacobi es defineix $\theta\colon \RR_{>0}\to \RR$
\[
\theta(x)=\sum_{n\in\ZZ}e^{-n^2\pi x}
\]
I definim també la funció $\omega$ com
\[
\omega(x)=\sum_{n\geq1}e^{-n^2\pi} x=\frac{\theta(x)-1}{2}
\]
}

\dfe{La funció $\Gamma$}{La funció $\Gamma$ es defineix:
\[
\Gamma(s)=\int_0^\infty e^{-x}x^s\frac{\dd x}{x}\qquad \text{per }\sigma>0
\]}
\dfe{Funció de Riemann completada}{Definim la funció de Riemann completada com
\[
\xi(s)\vcentcolon=\frac{1}{2}s(s-1)\pi^{-\frac{s}{2}}\Gamma\left(\frac{s}{2}\right)\zeta(s)
\]
I també tenim
\[
\frac{s(s-1)}{2}\int_1^\infty \left(x^{\frac{s}{2}-1}+x^{-\frac{s+1}{2}}\right)\omega(x)\dd x+\frac{1}{2}
\]}
\section{Productes d'Hadamard}
\dfe{Ordre d'una funció}{
Sigui $f(s)$ una funció entera. Es diu que $f$ és d'ordre menor o igual que $\alpha\in\RR_{\geq0}$ si existeix $r_0\in\RR_{\geq0}$ tal que
\[
f(s)=\bigO\left(e^{|s|^{\alpha}}\right)\qquad\text{per tot }|s|\geq r_0
\]
Aleshores l'ordre de $f$ es defineix
\[
\inf\{\alpha\in\RR_{\geq0}|f\text{ és d'ordre }\leq \alpha\}.
\]
}
\newpage

\newpage
\chapter{Enunciat sense demostració del criteri de sumació d’Abel (Teorema 4,capítol 1)}
\teorema{Criteri de sumació d'Abel}{\label{2.04_teorema_sumació_Abel}Sigui $a\colon\NN\rightarrow\CC$, considerem $A(t)\coloneq\sum_{n\leq t}a(n)$ les sumes parcials de $a$; i una funció $g:\RR_{\geq0}\rightarrow\CC$ amb derivada contínua en un interval $[x,y]\neq\varnothing$\footnote{Per alguna raó, al professor li ha agradat considerar l'interval $[y,x]$, però em nego a fer servir aquesta notació.}. Aleshores 
\[
\sum_{x<n\leq y}a(n)g(n)=A(y)g(y)-A(x)g(x)-\int_x^yA(t)g'(t)\dd t
\]
}

\newpage
\chapter{Existència del semiplà de convergència: \textbf{demostració} del Lema 5, $\S 1$; i enunciat i demostració del lema 6, $\S1$.}
\lema{Fita tècnica}{\label{2.05_lema_fita_de_sèries_de_Dirichlet} Suposem que $\sum_{n\geq 1} \frac{f(n)}{n^{s_0}}$ té sumes parcials fitades en $s_0\in\CC$, és a dir: existeix una fita $M\in\RR_{\geq 0}$ tal que $\left|\sum_{n\leq x}\frac{f(n)}{n^s_0}\right|\leq M$ per $x\geq 1$, aleshores per $s\in\CC$ tal que $\sigma>\sigma_0$ tenim:
\[
\left|\sum_{a< n\leq b}\frac{f(n)}{n^s}\right|\le2Ma^{\sigma_0-\sigma}\left(1+\frac{|s-s_0|}{\sigma-\sigma_0}\right)
\]
}
\pf{
\begin{enumerate}
    \ii Abel amb $a=\frac{f(n)}{n^{s_0}}$ i $g=t^{s_0-s}$.\\
    \ii ens queda això: $ Mb^{s_0-s}+Ma^{s_0-s}+M|s_0-s|\int_{a}^b\left|t^{s_0-s-1}\right|\dd t$
    \ii I d'aquí ja es pot veure com s'acaba.
\end{enumerate}
}
\lema{Semiplà de convergència}{
\label{2.06_lema_semipla de convergencia} Donada una sèrie de Dirichlet $\sum_{n\geq1}\frac{f(n)}{n^s}$, aleshores aquesta convergeix si i només si $\sigma>\sigma_c$ per alguna $\sigma_c\in\RR\cup\{\pm\infty\}$. De la mateixa manera que abans, aquesta $\sigma_c$ rep un nom: es diu $\sigma$ de convergència.
}
\pf{
\begin{enumerate}
    \ii apliquem el lema anterior.
\end{enumerate}
}

\newpage
\chapter{Enunciat i demostració del teorema de Landau (Teorema 13, capítol 1).}
\teorema{Landau}{\label{2.13_teorema_landau}Sigui $F(s)=\sum_{n\geq1}\frac{f(n)}{n^S}$ una sèrie de Dirichlet que convergeix per $\sigma>c\in\RR$. Si a més, tenim:
\begin{enumerate}
    \ii Que $f(n)\in\RR_{\geq0}$ per tota $n\geq n_0$,\\
    \ii i suposem que $F(s)$ estén a una funció holomorfa en un disc al voltant de $c$.
\end{enumerate}
Aleshores, $F(s)$ convergeix per $\s>c-\varepsilon$ per alguna $\varepsilon>0$.
}
\begin{figure}
\begin{center}
\begin{tikzpicture}
    \filldraw[color=black!10!white] (0,-1.3)--(0,1.3)--(2.5,1.3)--(2.5,-1.3)--(0,-1.3);
    \filldraw[color=black!10!white] (0,0) circle (0.5);
    \draw (-1,0)--(2.5,0);
    \draw (0,-1.3)--(0,1.3);
    \draw (0,0) circle (0.5);
    \filldraw (0,0) circle (0.05);
    \draw (-0.2,0.2) node {$c$};
    \draw (1,0) circle (1);
    \draw (1.2,0.25) node {$c+1$};
    \filldraw (1,0) circle (0.05);
    \draw[color=black!40!green] (1,0) circle (1.1);
\end{tikzpicture}
\caption{\label{fig:thm13}La zona pintada de gris és la zona en que $F$ és holomorfa.}
\end{center}
\end{figure}
\pf{
\begin{enumerate}
    \ii Fem sèrie de potències al voltant de $c+1$. Per hipòtesi $R>1$.\\
    \ii $F^{(k)}(s)=(-1)^k\sum_{n\geq1}\frac{f(n)}{n^s}(\log(n))^k$.\\
    \ii Sabem que podem reordenar, ja que tenim: $f(n)>0$ i la sèrie convergeix $\implies$ convergència absoluta $\implies$ podem reordenar.\\
    \ii Sabem que $F(c-\eps)$, (ja que $R>1$), i per tant, hem d'aconseguir posar-la com a sèries de potències.
    \ii fem aparèixer la sèrie de l'exponencial.\\
    \ii Ens queda: $\sum_{n\geq1}\frac{f(n)}{n^{c-\eps}}$.
\end{enumerate}
}
\newpage
\chapter{\textbf{Demostració} de la Proposició 2,$\S2$.}
\Prop{Extensió de caràcters de subgrups}{\label{3.02_proposició extensió de caràcters de subgrups}Sigui $H \leq G$ subgrup. Tot caràcter de $H$ estén a un caràcter de $G$.}
\pf{
\begin{enumerate}
    \ii Cas 1: $H=G$.\\
    \ii Cas 2: veure que un caràcter $\psi\colon H\to \CC^\times$ estén a un caràcter $\psi'\colon \langle H,x\rangle\to \CC ^\times$.\\
    \ii Com que $G$ és abelià, podem escriure els elements de $H'\vcentcolon=\langle H,x\rangle$ com $hx^a$. Com que $G$ és finit, $x$ té ordre finit, i per tant, sigui $n=\mathrm{ord}(x)$.\\
    \ii Sigui $\omega$ tal que $\psi(x^n)=\omega^n$\\
    \ii Definim $\psi'(hx^a)=\psi(h)\omega^a$.\\
    \ii Hem de veure que està ben definida.\\
    \ii Hem de veure que és un morfisme.\\
\end{enumerate}
} 
\chapter{\textbf{Demostració} del Teorema 11, $\S2$ (es poden assumir els lemes 7, 8 i la proposició 10).}
\lema{Convergència de les funcions $L$ de caràcters no principals}{\label{3.07_lema convergencia de funcions L(X,s), on X no és principal}Si $\chi \neq \chi_0$, llavors $L(\chi, s)$ té
\begin{itemize}
    \item[(i)] $\sigma_c = 0$, per tant $L(\chi, s)$ és holomorfa per $\sigma > 0$.
    \item[(ii)] $\sigma_a > 1$, per tant té una factorització 
    \[
        L(\chi, s) = \prod_p \frac{1}{1- \frac{\chi(p)}{p^{-s}}},
    \] atès que $\chi$ és completament multiplicativa. En particular, no s'anul·la per $\sigma > 1$. 
\end{itemize}}
\lema{Continuació meromorfa de $L(\chi_0,s)$}{\label{3.08_lema cont meromorfa de X_0}Si $\chi_0$ és el caràcter principal, aleshores:
\[
L(\chi,s)=\zeta(s)\prod_{p|m}\left(1-\frac{1}{p^s}\right)\qquad \text{per }\sigma>1.
\]
En particular $L(\chi,s)$ té continuació meromorfa en el semiplà a la dreta del 0 amb un únic pol simple en $s=1$.
}
\Prop{Producte d'Euler, i sèrie de Dirichlet de $\zeta_m$.}{\label{3.10_proposició producte d'Euler i sèrie de zeta_m}
Es té:
\begin{enumerate}
    \ii $\zeta_m(s)=\prod_{p\,\not\mid \,m}(1-p^{-f_ps})^{-\frac{\varphi(m)}{f_p}}$ per $\s>1$; i on $f_p$ és l'ordre de $p$ a $(\ZZ/m\ZZ)^\times$.\\
    \ii $\zeta_m(s)$ admet una expressió com a sèrie de Dirichlet amb coeficients a $\ZZ\geq0$ (en particular, $\RR_{\geq0}$) per a $\sigma>1$.    
\end{enumerate}
}
\teorema{Les funcions $L(\chi,s)$ no s'anul·len en $s=1$}{\label{3.11_teorema les funcions L no s'anul·len a l'1}\begin{enumerate}
    \ii La funció $\zeta_m(s)$ té un pol simple a $s=1$.\\
    \ii $L(\chi,1)\neq0$ si $\chi\neq\chi_0$.
\end{enumerate}}
\pf{
\begin{enumerate}
    \ii Només fa falta veure el primer apartat. (veure perquè)\\
    \ii Suposem que hi ha alguna $L(\chi,1)=0$. Aleshores, $\zeta_m(s)$ ha de ser holomorfa per $\s>0$. I per tant, com que els coeficients són enters posistius, per Landau tenim que la sèrie ha de convergir per $\s>0$.
    \ii Escriure la sèrie com a la prop 6.1: $\prod_{p|m}(1-p^{-f_ps})^{-\frac{\varphi(m)}{f_p}}$.\\
    \ii Sèrie geomètrica.\\
    \ii Entrar l'exponent a cada terme.\\
    \ii Simplificar fins a arribar a $\sum\frac{1}{n^{s\phi(m)}}$ que no convergeix per $s=\frac{1}{\varphi(m)}$. Que contradiu el que hem vist de que la convergència arriba fins al $s=0$.
\end{enumerate}
}
\newpage
\chapter{Demostració de de la Vallée-Poussin que $L(\chi, 1)\neq 0$ si $\chi$ és un caràcter de Dirichlet real no principal.}
\Prop{$L(\chi,1)$ no s'anul·la si $\chi$ és real no principal.}{Si $\chi$ és un caràcter complex, aleshores $L(\chi,1)\neq 0$.}
\pf{
\begin{enumerate}
    \ii Considerem $\psi(s)=\frac{L(\chi,s)L(\chi_0,s)}{L(\chi,2s)}$.\\
    \ii raonar que el numerador de $\psi$ és holomorf per $\s>0$.\\
    \ii Calcular $\lim_{\substack{s\to\frac{1}{2}\\s\in\RR_{>\frac{1}{2}}}}\psi(s)=0$.\\
    \ii Expandim la definició de funcions $L$ per $\s>1$ (que sabem que hi ha convergència absoluta).\\
    \ii Ens queda $\prod_{p\nmid m}\frac{1+p^{-s}}{1-p^{-s}}$.\\
    \ii Sèrie geomètrica.\\
    \ii Definir $\sum_{n\geq1}\frac{a_n}{n^s}=\prod(1+p^{-s})(1+p^{-s}+p^{-2s}+\dots)=\psi(s)$.\\
    \ii Fer sèries de potències al voltant de $s=2$, i com que sabem que $\psi$ és holomorfa per $\sigma>\frac{1}{2}$, tenim $R\geq\frac{3}{2}$.\\
    \ii Definim $b_k=\sum_{n\geq1}\frac{a_n(\log(n))^k}{n^2}$ (és $\psi^{(k)}(2)$ sense els signes).\\
    \ii Tornem a calcular el límit d'abans en un entorn de $s=\frac{1}{2}$, i ens surt que el límit ha de ser $\geq1$.
\end{enumerate}
}
\chapter{Enunciat i demostració del Lema 3, $\S3$.}
\lema{La funció $\Gamma$ és holomorfa.}{La funció $\Gamma(s)$ és holomorfa per $\sigma>0$.}
\pf{
\begin{enumerate}
    \ii Definim $\Gamma_n\vcentcolon=\int_{\frac{1}{n}}^{n}e^{-x}x^s\frac{\dd x}{x}$.\\
    \ii $\Gamma_n$ és holomorfa: utilitzem el següen teorema: Si $F\colon \{\s>0\}\times[\frac{1}{n},n]\to \CC$ tal que $F(s,x_0)$ és holomorfa i $F$ és contínua sobre $\{\s>0\}\times[\frac{1}{n},n]$; aleshores $\int_\frac{1}{n}^n F(s,x)\dd x$ és holomorfa.\\
    \ii Separem $|\Gamma(s)-\Gamma_n(s)|\leq|I_0|+|I_1|$, on $I_0$ i $I_1$ són les integrals dels costats. Volem veure que convergeixen uniformement sobre tots els compactes.\\
    \ii Agafem els rectangles $K(m,N,R)$, i veiem que convergeixen uniformement.\\
    \ii $|I_0|<\frac{1}{mn^{-m}}$; i $|I_1|<2Ce^{-\frac{n}{2}}$.
\end{enumerate}
}
\chapter{Demostració del Teorema 8, §3 (sense la part final de veure que la integral $I(s)$ defineix una funció holomorfa ).}
\teorema{}{Per $\sigma>1$, tenim:
\[
\pi^{-\frac{s}{2}}\Gamma\left(\frac{s}{2}\right)\zeta(s)=-\frac{1}{s}+\frac{1}{s-1}+\underbrace{\int_1^\infty\left(x^{\frac{s}{2}-1}+x^{-\frac{s}{2}-\frac{1}{2}}\right)\omega(x)\dd x}_{=I(s)}
\]
On la funció $I(s)$ és una funció holomorfa a tot $\CC$.
}
\pf{
\begin{enumerate}
    \ii Comencem amb $\Gamma(s)$, i apliquem el canvi de variable $x=n^2\pi x$ i ho treiem tot a fora.\\
    \ii Suposem que hi ha convergència absoluta, i fem el sumatori sobre $n\geq1$ dels 2 costats (en un costat apareix $\omega$ i a l'altre apareix $\zeta$).\\
    \ii Separem la integral $\int_0^\infty=\int_0^1+\int_{1}^{\infty}$.\\
    \ii Fem el canvi de varible $u=\frac{1}{t}$ a la integral de la esquerra.\\
    \ii Fem servir $\theta(\frac{1}{x})=\sqrt{x}\theta(x)$ i que $2\omega(x)+1=\theta$.\\
\end{enumerate}
}

\end{document}