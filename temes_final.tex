\documentclass{report}
% Configuración
%----------------
%   Importaciones
%----------------
\usepackage{xparse,xcolor}
\usepackage{sectsty}
\usepackage{amsmath,amssymb,latexsym,cancel,amsthm,mathtools} %amsfonts
\usepackage{tikz,pgfplots}
\pgfplotsset{compat=1.18, width=10cm}
\usepackage{multicol}
\usepackage{tikz-cd}
\usepackage{enumitem}

\usepackage[colorlinks=true]{hyperref}
\usepackage[most]{tcolorbox}
\usepackage{blindtext}
\usepackage{framed}
\usepackage{titletoc}
\usepackage{etoolbox}

\usepackage[explicit]{titlesec}
\usepackage{anyfontsize}

\usepackage[a4paper]{geometry}
\usepackage{graphicx, wrapfig, subcaption, setspace, booktabs}
\usepackage[T1]{fontenc}
\usepackage[catalan]{babel}
\usepackage[scaled]{helvet}     % Fuente del documento


\renewcommand{\familydefault}{\rmdefault} % per tornar a canviar la font posar aquí \sfdefault
\usepackage[utf8]{inputenc}
\usepackage{url, lipsum}
\usepackage{tabularx}
\usepackage{cancel}

% \setlength{\parindent}{0cm}
% \setlength{\parskip}{5pt}

%-----------------
%   Colores
%-----------------
\definecolor{mytheorembg}{HTML}{F2F2F9}
\definecolor{mylenmabg}{HTML}{FFFAF8}
\definecolor{mylenmafr}{HTML}{983b0f}
\definecolor{mypropbg}{HTML}{f2fbfc}
\definecolor{mypropfr}{HTML}{191971}
\definecolor{myp}{RGB}{197, 92, 212}
\definecolor{primary}{HTML}{207ba5}    % Color principal
\definecolor{migris}{RGB}{17, 17, 17}
\definecolor{grisfondo}{RGB}{249, 249, 249}
\definecolor{MyGrey}{HTML}{5B5B5B}


%----------------
%   Colores para
%   Las urls
%----------------
\hypersetup{
    colorlinks=true,
    linkcolor=black,
    filecolor=magenta,
    urlcolor=blue,
}

%----------------
%   Cajas
%----------------


% \newcommand{\comb}[2]{\begin{pmatrix}
%      #1\\
%      #2
% \end{pmatrix}}

\newcommand\fancybox[3]{%
    \tcbset{
        mybox/.style={
                enhanced,
                boxsep=0mm,
                opacityfill=0,
                overlay={
                        \coordinate (X) at ([xshift=-1mm, yshift=-1.5mm]frame.north west);
                        \node[align=right, text=#1, text width=2.5cm, anchor=north east] at (X) {\bf#2};
                        \draw[line width=0.5mm, color=#1] (frame.north west) -- (frame.south west);
                    }
            }
    }
    \begin{tcolorbox}[mybox]
        #3
    \end{tcolorbox}
}

\tcbuselibrary{theorems,skins,hooks}
\NewDocumentCommand\caja{m O{\Large #1} O{grisfondo} O{primary} O{number within=chapter}}
{
    \newtcbtheorem[#5]{#1}{\large #2}
    {%
        enhanced
        ,breakable
        ,colback = #3
        ,frame hidden
        ,boxrule = 0sp
        ,borderline west = {2pt}{0pt}{#4}
        ,sharp corners
        ,detach title
        ,before upper = \tcbtitle\par\smallskip
        ,coltitle = #4
        ,fonttitle = \bfseries%
        % ,description font = \mdseries
        ,separator sign none
        ,segmentation style={solid, #4}
    }
    {th}
}

\caja{Corolari}[Corol·lari][myp!10][myp!85!black]
\caja{Lema}[Lema][mylenmabg][mylenmafr]
\caja{Propo}[Proposició][mypropbg][mypropfr]
\caja{defi}[Definició][primary!12][primary]
\caja{Teorema}[Teorema][primary!12][primary]
\caja{Nota}[Nota][white][migris][no counter]
\caja{Propietats}[Propietats][white][migris][no counter]
\caja{preg}[Pregunta][white][migris][no counter]
\caja{Claim}[Claim:][white][migris][no counter]

%---------------
%   Comandos
%---------------
\newcommand{\teorema}[2]{\begin{Teorema}{#1}{}#2\end{Teorema}}
\newcommand{\pregunta}[2]{\begin{preg}{#1}{}#2\end{preg}}
\newcommand{\coro}[2]{\begin{Corolari}{#1}{}#2\end{Corolari}}
\newcommand{\lema}[2]{\begin{Lema}{#1}{}#2\end{Lema}}
\newcommand{\Prop}[2]{\begin{Propo}{#1}{}#2\end{Propo}}
\newcommand{\propietats}[2]{\begin{Propietats}{#1}{}#2\end{Propietats}}
\newcommand{\nota}[2]{\begin{Nota}{#1}{}{\em\color{MyGrey}#2}\end{Nota}}
\newcommand{\dfe}[2]{\begin{defi}{#1}{}#2\end{defi}}
\newcommand{\pf}[1]{\begin{proof}[\color{primary}\textbf{Demostració.}] #1 \end{proof}\vspace{7mm}}
\newcommand{\claim}[1]{\begin{Claim}{}{} #1 \end{Claim}}

\theoremstyle{definition}
\newtheorem*{exem}{\color{primary}Exemple}
\newcommand{\exemple}[1]{\begin{exem}#1\end{exem}}

\theoremstyle{definition}
\newtheorem*{solu}{\color{primary}Solució}
\newcommand{\sol}[1]{\begin{solu}#1\end{solu}}

\theoremstyle{definition}
\newtheorem*{obser}{\color{primary}Observació}
\newcommand{\obs}[1]{\begin{obser}#1\end{obser}}

%---------------
%   Listas
%---------------

% \newcommand{\cnumero}[2]{
%     \tikz[baseline=(myanchor.base)]
%     \node[minimum size=0.2cm,circle,
%         inner sep=1pt,draw, #2,thick,fill=#2](myanchor)
%     {\color{white}\bfseries\fontsize{8}{8}#1};}

\newcommand*{\itembolasazules}[1]{\protect\cnumero{#1}{primary}}
    
\newcommand{\listo}[1]{
    \begin{enumerate}[label=\itembolasazules{\arabic*}]
        #1
    \end{enumerate}
}

\newcommand{\listu}[1]{
    \begin{itemize}[label=$\color{primary} \bullet$]
        #1
    \end{itemize}
}

%-------------------------
% Tabla de Contenidos
%-------------------------

\patchcmd{\tableofcontents}{\contentsname}{\contentsname}{}{}

\renewenvironment{leftbar}
{\def\FrameCommand{\hspace{6em}%
        {\color{primary}\vrule width 2pt depth 6pt}\hspace{1em}}%
    \MakeFramed{\parshape 1 0cm \dimexpr\textwidth-6em\relax\FrameRestore}\vskip2pt%
}
{\endMakeFramed}

\titlecontents{chapter}[0em]
{\vspace*{2\baselineskip}}
{\parbox{4.5em}{%
        \hfill\Huge\bfseries\color{primary}\thecontentslabel}
    \vspace*{-2.3\baselineskip}\leftbar\textbf{\color{primary}\small\chaptername~\thecontentslabel}\\
}{}{\endleftbar}

\titlecontents{section}[8.4em]
{\contentslabel{3em}}{}{} 
{\hspace{0.5em}\nobreak\itshape\color{primary}\contentspage}

\titlecontents{subsection}[8.4em]
{\contentslabel{3em}}{}{} 
{\hspace{0.5em}\nobreak\itshape\color{primary}\contentspage}

%-----------------------------
%   Formato de los capitulos
%-----------------------------

%==================
% Capitulos
%==================
\newtcolorbox{titlecolorbox}[1]{ %the box around chapter
    coltext=white,
    colframe=primary,
    colback=primary,
    boxrule=0pt,
    arc=0pt,
    notitle,
    width=4.8em,
    height=2.4ex,
    before=\hfill
}


\makeatletter
\let\old@rule\@rule
\def\@rule[#1]#2#3{\textcolor{primary}{\old@rule[#1]{#2}{#3}}}
\makeatother

\titleformat{\chapter}[display]
{\Huge}
{}
{0pt}
{\begin{titlecolorbox}{}
        {\large\MakeUppercase{\bf\chaptername}}
    \end{titlecolorbox}
    \vspace*{-4.19ex}\noindent\rule{\textwidth}{0.4pt}
    \parbox[b]{\dimexpr\textwidth-4.8em\relax}{\raggedright\MakeUppercase{#1}}{\hfill\fontsize{70}{60}\selectfont{\color{primary}\thechapter}}
}
[]

\titleformat{name=\chapter,numberless}[display]
{\Huge\normalfont}
{}
{0pt}
{
    \vspace*{-4.19ex}\noindent\rule{\textwidth}{0.4pt}
    \parbox[b]{\dimexpr\textwidth-4.8em\relax}{\raggedright\MakeUppercase{#1}}
}
[]

%==============
% Secciones
%==============

\titleformat{\section}[hang]{\Large\normalfont\bfseries}
{\rlap{\color{primary}\rule[-6pt]{\textwidth}{0.4pt}}\colorbox{primary}{%
        \raisebox{0pt}[13pt][3pt]{ \makebox[60pt]{% height, width
                \fontfamily{cmr}\selectfont\color{white}{\thesection}}
        }}}%
{15pt}%
{ \color{primary}#1
    %
}
\titlespacing*{\section}{0pt}{3mm}{5mm}

%================
% Sub secciones
%================
\subsectionfont{\Large\color{primary}}

%---------------------
% Portada
%---------------------
\usetikzlibrary{ shapes.geometric }
\usetikzlibrary{calc}
\newcommand{\portada}[3]{
    \begin{tikzpicture}[remember picture,overlay]
        %%%%%%%%%%%%%%%%%%%% Background %%%%%%%%%%%%%%%%%%%%%%%%
        %\fill[primary] (current page.south west) rectangle (current page.north east);


        \foreach \i in {2.5,...,22}
            {
                \node[rounded corners,black!40,draw,regular polygon,regular polygon sides=6, minimum size=\i cm,ultra thick] at ($(current page.west)+(2.5,-5)$) {} ;
            }

        %%%%%%%%%%%%%%%%%%%% Background Polygon %%%%%%%%%%%%%%%%%%%% 
        \foreach \i in {0.5,...,22}
            {
                \node[rounded corners,black!40,draw,regular polygon,regular polygon sides=6, minimum size=\i cm,ultra thick] at ($(current page.north west)+(2.5,0)$) {} ;
            }

        \foreach \i in {0.5,...,22}
            {
                \node[rounded corners,black!20,draw,regular polygon,regular polygon sides=6, minimum size=\i cm,ultra thick] at ($(current page.north east)+(0,-9.5)$) {} ;
            }


        \foreach \i in {21,...,6}
            {
                \node[black!85,rounded corners,draw,regular polygon,regular polygon sides=6, minimum size=\i cm,ultra thick] at ($(current page.south east)+(-0.2,-0.45)$) {} ;
            }


        %%%%%%%%%%%%%%%%%%%% Title of the Report %%%%%%%%%%%%%%%%%%%% 
        \node[left,black!70,minimum width=0.625*\paperwidth,minimum height=3cm, rounded corners] at ($(current page.north east)+(0,-9.5)$)
        {
            {\fontsize{25}{30} \selectfont \bfseries #1} 
        };

        %%%%%%%%%%%%%%%%%%%% Subtitle %%%%%%%%%%%%%%%%%%%% 
        \node[left,black!60,minimum width=0.625*\paperwidth,minimum height=2cm, rounded corners] at ($(current page.north east)+(0,-11)$)
        {
            {\huge \textit{#2}}
        };

        %%%%%%%%%%%%%%%%%%%% Author Name %%%%%%%%%%%%%%%%%%%% 
        \node[left,black!60,minimum width=0.625*\paperwidth,minimum height=2cm, rounded corners] at ($(current page.north east)+(0,-13)$)
        {
            {\Large \textsc{#3}}
        };

        %%%%%%%%%%%%%%%%%%%% Year %%%%%%%%%%%%%%%%%%%% 
        \node[rounded corners,fill=black!70,text =primary!5,regular polygon,regular polygon sides=6, minimum size=2.5 cm,inner sep=0,ultra thick] at ($(current page.west)+(2.5,-5)$) {\LARGE \bfseries \the\year{}};

    \end{tikzpicture}
}

% declaracions de dreceres
\providecommand{\ol}{\overline}
\providecommand{\ul}{\underline}
\providecommand{\wt}{\widetilde}
\providecommand{\wh}{\widehat}
\providecommand{\eps}{\varepsilon}
\providecommand{\half}{\frac{1}{2}}
\providecommand{\inv}{^{-1}}
\newcommand{\dang}{\measuredangle} %% Directed angle
\providecommand{\CC}{\mathbb C}
\providecommand{\s}{\sigma}
\providecommand{\FF}{\mathbb F}
\providecommand{\KK}{\mathbb K}
\providecommand{\NN}{{\mathbb Z_{\geq1}}}
\providecommand{\QQ}{\mathbb Q}
\providecommand{\RR}{\mathbb R}
\providecommand{\ZZ}{\mathbb Z}
\providecommand{\PP}{\mathbb P}
\providecommand{\bigO}{\mathcal O}
\newcommand\smallO{
  \mathchoice
    {{\scriptstyle\mathcal{O}}}% \displaystyle
    {{\scriptstyle\mathcal{O}}}% \textstyle
    {{\scriptscriptstyle\mathcal{O}}}% \scriptstyle
    {\scalebox{.6}{$\scriptscriptstyle\mathcal{O}$}}%\scriptscriptstyle
  }
\providecommand{\dd}{\mathrm{d}}

\providecommand{\ts}{\textsuperscript}
\providecommand{\dg}{^\circ}
\providecommand{\ii}{\item}
\DeclareMathOperator*{\lcm}{lcm}
\DeclareMathOperator*{\argmin}{arg min}
\DeclareMathOperator*{\argmax}{arg max}
\DeclareMathOperator{\Ima}{Im}
\DeclareMathOperator{\Hom}{Hom}
\newcommand{\mcal}[1]{\mathcal{#1}}

\begin{document}
\renewcommand{\rmdefault}{cmr}

\pagestyle{empty}
\portada{Teoria pel final de mètodes analítics en teoria de nombres}{Curs 2025-2026}{Bernat Esteve}
\newpage

\tableofcontents

\newpage
\chapter{Definicions}
\section{Sèries de Dirichlet}
\dfe{Sèrie de Dirichlet}{Una sèrie de Dirichlet és una sèrie de la forma
\[
\sum_{n\in\NN}\frac{f(n)}{n^s}
\]
On $f\colon\NN\rightarrow\CC$ és una funció aritmètica, i $s\in\CC$.
}

\dfe{Convergència de productoris}{
Sigui $\{z_n\}_n\subset\CC$ una successió de complexos, aleshores el productori $\prod z_n$ convergeix si i només si existeix el límit $\lim_{n\rightarrow\infty}\prod_{i=1}^nz_n$, i aquest és no nul.
}

\dfe{Convergència absoluta de productoris}{
Donada una seqüència $z_n$ amb $\Re(z_n)>0$, aleshores el productori $\prod z_r$ es diu que convergeix absolutament si $\sum \log(z_r)$ convergeix absolutament.
}
\section{Funcions L}
\dfe{Caràcter d'un grup}{
Sigui $G$ un grup finit i abelià, aleshores un caràcter de $G$ serà un morfisme $\psi\colon G\rightarrow \CC^*$ (on $\CC^*$ és el grup multiplicatiu de $\CC\setminus\{0\}$)}

\dfe{El grup de caràcters}{Denotem per $\widehat{G}$ al grup $\widehat{G}=\Hom(G,\CC^*)=\{\text{Caràcters de G}\}$, on la operació és:
\[
\psi,\phi\colon G\rightarrow\CC^*\qquad\text{aleshores }(\psi{\;\cdot_{\widehat{G}} } \;\phi)(g)\mapsto\psi(g)\cdot_{\small{\mathbb{C}}^*}\phi(g)
\]
I anomenarem a $\widehat{G}$ el grup de caràcters de $G$.
}

\dfe{Caràcter mòdul $m$}{
Un caràcter mòdul $m$ és un caràcter de $(\mathbb{Z}/m\mathbb{Z})^\times$.}

\dfe{Caràcter principal mòdul $m$}{El caràcter principal mòdul $m$ és $\chi_0(a) \colon \ZZ^+ \to \CC$ definit per 
\[
    \chi_0(a) = \begin{cases}
        1 & \gcd(a, m) = 1 \\
        0 & \text{altrament}
    \end{cases}
\]}

\dfe{Funció $L$ associada a un caràcter de Dirichlet}{La funció $L$ associada al caràcter de Dirichlet mòdul $m$ $\chi$ és la sèrie de Dirichlet \[
L(\chi, s) = \sum_{n\geq 1} \frac{\chi(n)}{n^s}.
\].}

\dfe{Funció $m$-èsima de Dirichlet}{Sigui $m\in\NN$, aleshores la funció $m$-èsima de Dirichlet es defineix com
\[
\zeta_m(s)\vcentcolon=\prod_\chi L(\chi,s)
\]
On el producte recorre tots els caràcters de Dirichlet mòdul $m$.
}

\dfe{Caràcters reals i complexos.}{Diem que $\chi\colon\NN\to \CC$ és un caràcter de Dirichlet real si $\Ima(\chi)\subset\RR$. És a dir $\Ima(\chi)\subset\{-1,0,1\}$. I direm que és complex altrament.}

\dfe{Caràcter conjugat}{Direm $\overline{X}$ al caràcter de Dirichlet conjugat
\[
\overline{X}\colon\NN\to\CC\qquad\text{que envia } a\in\ZZ\text{ a }\overline{\chi}(a)=\overline{\chi(a)}
\]
I no és massa difícil de veure que aquesta funció és un caràcter de Dirichlet del mateix mòdul.}
\section{Continuació meromorfa i equació funcional de $\zeta(s)$}

\dfe{$\theta$ de Jacobi}{
La funció $\theta$ de Jacobi es defineix $\theta\colon \RR_{>0}\to \RR$
\[
\theta(x)=\sum_{n\in\ZZ}e^{-n^2\pi x}
\]
I definim també la funció $\omega$ com
\[
\omega(x)=\sum_{n\geq1}e^{-n^2\pi x}=\frac{\theta(x)-1}{2}
\]
}

\dfe{La funció $\Gamma$}{La funció $\Gamma$ es defineix:
\[
\Gamma(s)=\int_0^\infty e^{-x}x^s\frac{\dd x}{x}\qquad \text{per }\sigma>0
\]}
\dfe{Funció de Riemann completada}{Definim la funció de Riemann completada com
\[
\xi(s)\vcentcolon=\frac{1}{2}s(s-1)\pi^{-\frac{s}{2}}\Gamma\left(\frac{s}{2}\right)\zeta(s)
\]
I també tenim
\[
\frac{s(s-1)}{2}\int_1^\infty \left(x^{\frac{s}{2}-1}+x^{-\frac{s+1}{2}}\right)\omega(x)\dd x+\frac{1}{2}
\]}
\section{Productes d'Hadamard}
\dfe{Ordre d'una funció}{
Sigui $f(s)$ una funció entera. Es diu que $f$ és d'ordre menor o igual que $\alpha\in\RR_{\geq0}$ si existeix $r_0\in\RR_{\geq0}$ tal que
\[
f(s)=\bigO\left(e^{|s|^{\alpha}}\right)\qquad\text{per tot }|s|\geq r_0
\]
Aleshores l'ordre de $f$ es defineix
\[
\inf\{\alpha\in\RR_{\geq0}|f\text{ és d'ordre }\leq \alpha\}.
\]
}
\newpage

\newpage
\chapter{Enunciat sense demostració del criteri de sumació d’Abel (Teorema 4,capítol 1)}
\teorema{Criteri de sumació d'Abel}{\label{2.04_teorema_sumació_Abel}Sigui $a\colon\NN\rightarrow\CC$, considerem $A(t)\coloneq\sum_{n\leq t}a(n)$ les sumes parcials de $a$; i una funció $g:\RR_{\geq0}\rightarrow\CC$ amb derivada contínua en un interval $[x,y]\neq\varnothing$\footnote{Per alguna raó, al professor li ha agradat considerar l'interval $[y,x]$, però em nego a fer servir aquesta notació.}. Aleshores 
\[
\sum_{x<n\leq y}a(n)g(n)=A(y)g(y)-A(x)g(x)-\int_x^yA(t)g'(t)\dd t
\]
}

\newpage
\chapter{Demostració del Teorema 8, capítol 3 (sense la part final de veure que la integral $I(s)$ defineix una funció holomorfa ).}
\teorema{}{Per $\sigma>1$, tenim:
\[
\pi^{-\frac{s}{2}}\Gamma\left(\frac{s}{2}\right)\zeta(s)=-\frac{1}{s}+\frac{1}{s-1}+\underbrace{\int_1^\infty\left(x^{\frac{s}{2}-1}+x^{-\frac{s}{2}-\frac{1}{2}}\right)\omega(x)\dd x}_{=I(s)}
\]
On la funció $I(s)$ és una funció holomorfa a tot $\CC$.
}
\pf{
\begin{enumerate}
    \ii Comencem amb $\Gamma(s)$, i apliquem el canvi de variable $x=n^2\pi x$ i ho treiem tot a fora.\\
    \ii Suposem que hi ha convergència absoluta, i fem el sumatori sobre $n\geq1$ dels 2 costats (en un costat apareix $\omega$ i a l'altre apareix $\zeta$).\\
    \ii Separem la integral $\int_0^\infty=\int_0^1+\int_{1}^{\infty}$.\\
    \ii Fem el canvi de varible $u=\frac{1}{t}$ a la integral de la esquerra.\\
    \ii Fem servir $\theta(\frac{1}{x})=\sqrt{x}\theta(x)$ i que $2\omega(x)+1=\theta$.\\
\end{enumerate}
}
\newpage
\chapter{Demostració de la Proposició 6 del capítol 4, donat el seu enunciat i assumint la Proposició 7.}
\Prop{proposició 6}{Sigui $f$ una funció entera d'ordre $\alpha$. I suposem que $f$ no té zeros. Aleshores es té que $f=e^{g(s)}$ on $g$ és un polinomi de grau $\alpha$. És a dir, que el grau d'una funció entera sense zeros ha de ser un enter.
}
\pf{
    Ens caldrà el següent resultat d'AnCo:
    \Prop{proposició 7}{Sigui $U\subseteq \CC$ simplement connex. Sigui $f:U\to\CC$ holomorfa i sense zeros en $U$. Aleshores existeix $g:U\to\CC$ holomorfa tal que $f(s)=e^{g(s)}$ per tot $s\in U$.}
    \begin{enumerate}
        \item Agafar la $g(s)$ que ens dona la proposició.
    \end{enumerate}
    Per la proposició, podem dir que $f(s)=e^{g(s)}$ amb $g$ entera. Com que $f$ té ordre $\alpha$:
    \[
        |f(s)|\leq K e^{|s|^{\alpha+\delta}}\quad\forall \delta>0,\quad \text{per }K\in\CC, |s|\gg1
    \]
    I d'aquí en podem treure que
    \[
        \Re(g(s))\leq |s|^{\alpha+\delta}+\log(K)\quad\forall \delta>0,\quad \text{per }K\in\CC, |s|\gg1
    \]
    I ara, engrandint el valor de $\delta$ si fa falta, tenim
    \[
        \Re(g(s))\leq 2|s|^{\alpha+\delta}        
    \]
    i considerant $g\to g-g(0)$ (és a dir que podem assumir que $g(0)=0$), podem expressar $g$ amb la seva sèrie de Taylor:
    \[
        g(s)=\sum_{n=1}a_ns^n
    \]
    Amb $a_n=\frac{g^{(n)}(0)}{n!}$, i ara utilitzant Cauchy:
    \begin{align*}
    a_n=&\frac{1}{2\pi i}\int_{|s|=\RR^n}\frac{g(s)}{s^{n+1}}\dd s=\\
    =&\frac{1}{2\pi i}\int_{0}^{2\pi}\frac{g(Re^{i\theta})}{R^{n+1}e^{(n+1)i\theta}} Re^{i\theta} i\dd \theta=\\
    =&\frac{1}{2\pi R^n}\int_{0}^{2\pi}\frac{g(Re^{i\theta})}{e^{ni\theta}}\dd \theta
    \end{align*}
    Considerem ara la següent integral
    \[
        \int_{|s|=R}g(s)s^{n-1}\dd s =\underbrace{\int_0^{2\pi}g(Re^{i\theta})R^n i e^{in\theta}\dd\theta}_{I_1}=0=\underbrace{\int_0^{2\pi}\ol{g(Re^{i\theta})}R^n i e^{-in\theta}\dd\theta}_{I_2}
    \]
    Això és zero degut a que estem considerant $n\geq1$, i per tant, la funció que estem integrant és entera. I per tant:
    \[
        a_n=\frac{1}{2\pi R^n}\int_0^{2\pi}(g(Re^{i\theta})+\ol{g(Re^{i\theta})})e^{-in\theta}\dd \theta=\frac{1}{\pi R^n}\int_0^{2\pi}\Re (g(Re^{i\theta})e^{-in\theta})\dd \theta
    \]
    Nosaltres sabem que $\Re(g(Re^{i\theta}))<2R^{\alpha+\delta}$; però podria ser que $\Re(g(Re^{i\theta}))<0$ fos molt negatiu; per tant, ens interessa tenir una fita per $|\Re(Re^{i\theta})|$:
    \[
        |a_n|\leq\frac{1}{\pi R^n}\int_0^{2\pi}|\Re(g(Re^{i\theta}))|
    \]
    Però notem que:
    \[
        g(0)=0=\int_{|s|=R}\frac{g(s)}s\dd s=\int_0^{2\pi}g(Re^{i\theta})\dd \theta=0
    \]
    I d'aquí en podem deduir que 
    \[
        \frac{1}{2\pi}\int_0^{2\pi}\Re(g(Re^{i\theta}))\dd \theta = 0
    \]
    Per tant, podem canviar el coeficient sense cap preocupació:
    \[
        \frac{1}{\pi R^n}\int_0^{2\pi}\Re(g(Re^{i\theta}))\dd \theta = 0
    \]
    Però abans hem vist que el valor absolut de l'integrand era una fita de $|a_n|$:
    \[
        |a_n|\leq \frac{1}{\pi R^n}\int_0^{2\pi}\left|\Re(g(Re^{i\theta}))\right|+\Re(g(Re^{i\theta}))\dd \theta\leq\frac{1}{\pi R^n}\int_0^{2\pi}\max\{0,2\Re(g(Re^{i\theta}))\}\dd\theta
    \]
    Però $\Re(g(s))\leq 2|s|^{\alpha+\delta}$, per tant (per $|s|\gg1$, però això no és un problema):
    \[
        |a_n|\leq \frac{1}{\pi R^n}\int_0^{2\pi}4R^{\alpha+\delta}\dd\theta=\frac{8\pi R^{\alpha+\delta}}{\pi R^n}=8R^{\alpha+\delta-n}
    \]
    I per $\alpha+\delta>n$, $|a_n|\leq 8R^{\alpha+\delta-n}\to 0$; per tant tots els coeficients de la sèrie de Taylor amb grau més gran que $\alpha$ seràn 0; i per tant, $g$ és un polinomi (de grau $\alpha$, ja que sabem que $f$ és d'ordre $\alpha$), que és el que volíem veure.
}
\newpage
\chapter{Demostració del Teorema 15 del capítol 4, donat el seu enunciat i assumint els corol·laris 10 i 14.}
\teorema{}{
    La funció zeta de Riemann completada
    \[
        \xi(s)=\half s(s-1)\pi^{-s/2}\zeta(s)\Gamma\left(\frac s2\right)
    \]
    satisfà:
    \begin{enumerate}
        \item Es una funció d'ordre 1.
        \item Té un nombre infinit de zeros (a la franja crítica).
        \item $\sum_{n\geq1}\frac{1}{|\rho_n|}$ divergeix; i $\sum_{n\geq1}\frac{1}{|\rho_n|^{1+\eps}}$ convergeix $\forall \eps>0$.
    \end{enumerate}
}
Notem que els zeros de la funció $\xi$ són exactament els zeros (amb multiplicitat) de $\zeta$ en la franja crítica.
\pf{
    I amb aquesta remarca, procedirem a demostrar els 3 apartats del teorema.\\
    Notem que pel corol·lari 9 del tema 3, tenim que la funció $\xi$ és una funció entera.
    A més, $\half(s-1)s\pi^{-s/2} = \bigO\left(e^{|s|^{1+\eps}}\right)$ per tot $\eps$ positiu, i $|s|\gg1$.\\
    Ara només fa falta tractar la funció $\Gamma$, però sabem que $\Gamma(s/2)=\bigO(e^{|s|^{1+\eps}})$ $\forall \eps>0$ i $|s|\gg1$ amb $\s\geq0$, tot i que aquesta última condició no ens importa, ja que $\xi(s)=\xi(1-s)$ i tot el que poguem dir per $\s>0$ ho podem dir també per $\s<0$.\\
    Però pel teorema 15 del capítol 1, per $\s>0$ tenim que
    \[
        \zeta(s)=\underbrace{\frac{1}{s-1}+1}_{\text{per $|s|>1$ fitat}}-s\underbrace{\int_{1}^{\infty}\frac{\{x\}}{x^{s+1}}\dd x}_{\text{per }|s|\gg1\text{ amb }\s>1/2\text{, fitat}}=\bigO(|s|).
    \]
    Per tant, per $s\gg1$, $\bigO(|s|)=\bigO\left(e^{|s|^\eps}\right)$, és a dir, que $\zeta(s)=\bigO(e^{|s|^\eps})$ per $s\gg 1$ i $\s\geq\half$.\\
    En resum $|\xi(s)|=\bigO(e^{|s|^{1+\eps}})$ $\forall\eps>0$, $|s|\gg1$ i $\s\geq\half$. I per l'equació funcional de $\xi$, ho tenim per $|s|\gg1$. És a dir, que $\xi$ és d'ordre $\leq$1. I per tant, ara ens fa falta veure que l'ordre és exactament 1.\\[4mm]
    Per veure que la funció assoleix valors grans, farem un claim:
    \claim{Sigui $r\in\NN$, aleshores $\log(r!)\sim r\log(r)$ quan $r\to \infty$.}
    \pf{La demostració de la claim consisteix en veure que
    \[
        \log(r^r)=r\log(r)>\log(r!)=\sum_{i=1}^r \log(i)>\int_1^r\log(x)\dd x=r\log(r)-r+1
    \]}
    O sigui, que en els enters, sabem que la funció $\Gamma$ és gran:
    \[
        \log(\xi)=\log(\Gamma)-\frac s2\log(\pi)+\log\left(\half s(s-1)\right)+\log(\zeta)
    \]
    Però en $s=2r$: sabem que $\log\Gamma(r)=\log(r-1)!\sim r\log r$. També sabem que $\log\zeta(2r)=\log\sum_{n\geq1}\frac{1}{n^{2r}}\xrightarrow{r\to\infty}0$ (ja que $\sum_{n\geq1}\frac{1}{n^r}\to 1$.). I finalment la resta de coses són de la mida $-r/2\log(\pi)$. Per tant
    \[
        \log(\xi)\sim r\log(r)\qquad\text{quan }r\to\infty.
    \]
    I d'aquí en podem deduir que $\xi$ té ordre 1. I d'aquí se'n desprenen la resta de coses.\\[5mm]
    Per tant, suposem que el número de zeros és finit, aleshores:
    \[
        \sum_{n\in N_f}\frac{1}{|\rho_n|}<\infty\quad\implies\quad|\xi(s)|<e^{C|s|}\quad\text{per }s\gg 1
    \]
    On la implicació ve del corol·lari 14; però això contradiu el fet que $\log(\xi(r))\sim r\log(r)$. I això contradiu el fet que el sumatori convergeix.\\[4mm]
    Per veure que el segon sumatori convergeix, en tenim prou en invocar el corol·lari 10, i el resultat és immediat.
}
\newpage
\chapter{Demostració del Teorema 5 del capítol 5. La demostració ha d'incloure els enunciats i demostracions de Lema 2 i), Lema 3, i Proposició 4 i).}
\lema{}{\label{6.02_lema}Per $\s>1$ es té:
\begin{enumerate}
    \ii $\Re\;\log(\zeta(s))=\sum_p\sum_{m\geq1}\frac{\cos(mt)\log p}{mp^{\s m}}$\\
\end{enumerate}
On en el segon apartat, la funció $\Lambda(n)=\left\{\begin{array}{cc}
    \log p&\text{ si }n=p^k\\
    0&\text{ altrament.}
\end{array}\right.$
}
\pf{
Per $s\in\RR_{>1}$ ja sabem del corol·lari $\ref{2.17_coro sèries de potències de p}$ del capítol \ref{capítol 1: sèries de Dirichlet}, que 
\[
    \log(\zeta(s))=-\sum_{p}\log(1-p^{-s})=\sum_p\sum_{m\geq 1}\frac{1}{mp^{\s m}}
\]
I per continuació analítica, la igualtat es certa per $s=\s+it\in \CC$. Per tant:
\[
    \log(\zeta(s))=\sum_p\sum_{m\geq1}\frac{1}{mp^{ms}}=\sum_p\sum_{m\geq1}\frac{p^{-im t}}{mp^{m\s}}
\]
I ara, si prenem la part real del que tenim:
\[
    \Re\,\log(\zeta(s))=\sum_p\sum_{m\geq1}\frac{\cos(mt\log p)}{mp^{m\s}}
\]
On la igualtat ve de que si $\theta\in\RR$, aleshores $p^{i\theta}=\log p(\cos(\theta)+i\sin(\theta))$.
Que és el que volíem veure.\\
}
\lema{Lema de Mertens}{\label{6.03_lema de Mertens} Per qualsevol $\theta\in\RR$, tenim la següent desigualtat.
\[
    3+4\cos(\theta)+\cos(2\theta)\geq0
\]
}
\pf{
Utilitzant la fórmula de l'angle doble del cosinus, tenim:
\[
    3+4\cos(\theta)+\cos(2\theta)=
    3+4\cos(\theta)+\cos^2(\theta)-\sin^2(\theta)=
    2(1+\cos(\theta))^2\geq0
\]
}
\Prop{Mertens per a la funció $\zeta$.}{\label{6.04_prop}
    Per $\s>1$, tenim:
    \begin{enumerate}
        \ii $\zeta^3(\s)|\zeta^4(\s+it)||\zeta(\s+i2t)|\geq1$,\\
    \end{enumerate}

}
\pf{
Com que a $\s>1$:
\[
    |\zeta(s)|=e^{\Re\log(\zeta(s))}
\]
Només hem de veure que:
\[
    0\leq 3\log(\zeta(\s))+4\Re\log(\zeta(\s+it))+\Re\log\zeta(\s+2it)
\]
Però pel lema \ref{6.02_lema}, tenim
\[
    3\log(\zeta(\s))+4\Re\log(\zeta(\s+it))+\Re\log\zeta(\s+2it)=\sum_{p}\sum_{m\geq1}\frac{\log(p)}{mp^{\s m}}(3+4\cos(mt\log p)+\cos(2mt\log p))\geq0
\]
Que és el que havíem de veure pel primer apartat.\\
}
\teorema{Regió lliure de zeros I}{\label{6.05_teorema regió lliure de zeros en sigma=0,1}La funció $\zeta(s)\neq0$ en $\s=1$ (i per la simetria de $\zeta$, en $\s=0$).}
\pf{
Procedirem per reducció a l'absurd: suposem que existeix alguna $t\in \RR$ tal que $\zeta(1+it)=0$. Clarament $t\neq0$, ja que $\zeta(1)\neq0$. De fet, podem dir una miqueta més: com que $\zeta(s)\sim \frac{1}{\s-1}$ quan $\s\to 1$.\\
Com que $\zeta$ té un zero en $1+it$, podem escriure $\zeta(s)=(s-(1+it))^mf(s)$ per alguna $m\geq1$, on ara $f$ no s'anul·la en $1+it$. Per tant, si considerem
\[
    \zeta(\s+it)=(\s-1)^m f(\s+it)
\]
I ara, utilitzem la proposició \ref{6.04_prop}:
\[
    \lim_{\s\to 1}\underbrace{\zeta^3(\s)}_{\sim (\s-1)^{-3}}
    \underbrace{|\zeta^4(\s+it)|}_{\sim(\s-1)^4|f(1+it)|}
    \underbrace{|\zeta(\s+2it)|}_{<\infty}=\lim_{\s\to 1}(\s-1)^{4m-3}|\dots|= 0
\]
Però, la proposició ens diu que això és com a mínim 1 i, per tant, contradicció.
}
\newpage
\chapter{Demostració del Teorema 8 §5, donat el seu enunciat. A la demostració cal ser capaç d'enunciar el Lema 6 i) i el Lema 7 de capítol 5 amb les seves dues parts (però no cal donar les demostracions dels lemes).}
\lema{Lema 6}{
    Per $1<\s\leq2$
    \[
        -\frac{\zeta'(s)}{\zeta(s)}<\frac{1}{\s-1}+K\qquad\text{on }K\in\RR
    \]
}
\lema{Lema 7}{
    Sigui $\rho=\beta+i\gamma$ un zero no trivial de $\zeta(s)$ amb $\gamma\geq2$. Sigui $s=\s+i t$ amb $1< \s\leq 2$ i $t\geq2$, aleshores
    \begin{enumerate}
        \item $-\Re\left(\frac{\zeta'(\s+2it)}{\zeta(\s+2it)}\right)<K\log(t)$ per $k\in\RR_{>0}$.
        \item Si a més $t=\gamma$, aleshores: $-\Re\left(\frac{\zeta'(\s+it)}{\zeta(\s+it)}\right)<k\log(t)-\frac{1}{\s-\beta}$ per $k\in\RR_{\geq0}$
    \end{enumerate}
}
\lema{Teorema 8}{
    Existeix $c\in\RR_{>0}$ tal que $\zeta$ no té zeros a la regió $t\geq2,$ i $\s\leq1-\frac{c}{\log(t)}$
}
\pf{
    Per la segona part 4, tenim que per $\s>1$:
    \begin{equation}
        3\left(-\frac{\zeta'(\s)}{\zeta(\s)}\right)-4\Re\left(\frac{\zeta'(\s+it)}{\zeta(\s+it)}\right)-\Re\left(\frac{\zeta'(\s+2it)}{\zeta(\s+2it)}\right)\geq0
    \end{equation}\label{eq_lema_4}
    Sigui $s(t)=\s+it$ on $\s=1+\frac{\delta}{\log(t)}$, per algun $\delta\in(0,\log2]$ (prenem $\delta$ en aquest interval per tal de tenir $\s\in(1,2]$, per a qualsevol $t\geq2$). Això ens donarà aquesta corba:
    \begin{center}
    \begin{tikzpicture}[domain=0:3, scale=0.5]
    \draw[ultra thick] (-0.3,0)--(6.3,0);
    \draw[ultra thick] (0,-0.3)--(0,4);
    \draw[ultra thick] (3,-0.3)--(3,4);
    \draw[ultra thick] (6,-0.3)--(6,4);
    \node at (0,4.3) {$\s=0$};
    \node at (3,4.3) {$\s=1$};
    \node at (6,4.3) {$\s=2$};
    \node at (7.5,0) {$t=0$};
    \node at (7.5,1) {$t=2$};
    \draw[red!60!black, domain = 2:8]   plot ({3+3*(ln(2)-.00)/(ln(\x))},    \x/2);
    \draw[red!90!black, domain = 2:8]   plot ({3+3*(ln(2)-.15)/(ln(\x))},    \x/2);
    \draw[orange, domain = 2:8]         plot ({3+3*(ln(2)-.30)/(ln(\x))},    \x/2);
    \draw[pink!70!purple, domain = 2:8] plot ({3+3*(ln(2)-.45)/(ln(\x))},    \x/2);
    \draw[pink, domain = 2:8]           plot ({3+3*(ln(2)-.60)/(ln(\x))},    \x/2);
    \filldraw (1.5,3) circle (0.1);
    \node at (1.5,3.4) {$\rho$};
    \draw [dashed] (0,3)--(6,3);
    \filldraw[red!60!black] ({3+3*ln(2)/(ln(6))},3) circle (0.1);
    \node[red!60!black] at ({3.2+3*ln(2)/(ln(6))},3.4) {$s$};
    \draw[ultra thick] (-0.3,1)--(6.3,1);
    \end{tikzpicture}
    \end{center}
    I ara, prenem $t$ com l'ordenada del zero no trivial de $\zeta$: $\rho=\beta+i\gamma$.\\
    Apliquem ara el lema 6 i 7 a la desigualtat del principi (\ref{eq_lema_4}), i obtenim:
    \[
        0\leq \underbrace{\frac{3}{\s-1}+k_1}_{\text{lema 6, i)}}+\underbrace{k_2\log(t)-\frac{4}{\s-\beta}}_{\text{lema 7, ii)}}+\underbrace{k_3\log(t)}_{\text{lema 7 i)}}
    \]
    Si prenem $k/2\geq\max\{k_1,k_2,k_3\}$, podem suposar que tenim una $k$ comuna (i si cal, la fem encara més gran per a quedar-nos només el termes dominants), obtenim:
    \[
        \frac{4}{\s-\beta}< \frac{3}{\s-1}+k\log(t)
    \]
    Posant $\s=1+\frac{\delta}{\log(t)}$:
    \[
        \frac{4}{1+\frac{\delta}{\log(t)}-\beta}< \frac{3\log(t)}{\delta}+k\log(t)
    \]
    I que manipulant una mica les coses, tenim:
    \[
        \frac{4}{1+\frac{\delta}{\log(t)}-\beta}< \frac{\log(t)}{\delta}\left(3+k\delta\right).
    \]
    Com que $\beta\in(0,1)$, sabem que $\s-\beta>0$, és a dir, que podem invertir les coses:
    \[
        \frac{4\delta}{\log(t)(3+k\delta)}<1+\frac{\delta}{\log(t)}-\beta
    \]
    És a dir:
    \[
        \beta<-\frac{4\delta}{\log(t)(3+k\delta)}+1+\frac{\delta}{\log(t)}=
        1-\frac{1}{\log(t)}\underbrace{\delta\left(\frac{4}{3+k\delta}-1\right)}_{=c}
    \]
    Per tant, per $\delta$ prou petita ($\delta<\frac{1}{k}$), tenim que la $c>0$, que és el que volíem veure.
}
\newpage
\chapter{Demostració de la Proposició 8 de capítol 6 assumint el Lema 9 (l'enunciat del qual cal recordar).}
\Prop{Proposició 8}{Per $T\gg1$, i $x\in\RR_{\geq2}$, i $c=1+\frac{1}{\log x}$, aleshores:
\[
    \frac{1}{2\pi i}\int_{c-iT}^{c+iT}\underbrace{\left(-\frac{\zeta'(s)}{\zeta(s)}\right)\frac{x^s}{s}}_{F(s)}\dd s=x-\sum_{\substack{\rho=\beta+i\gamma\\|\gamma|<T}}\frac{x^{\rho}}{\rho}-\frac{\zeta'(0)}{\zeta(0)}-\frac{1}{2}\log\left(1-\frac{1}{x^2}\right)+\bigO\left(\frac{x(\log T)^2}{T\log(x)}\right)
\]
On la suma sobre $\rho$ és una suma sobre els zeros no trivials de $\zeta$.}
\textit{Comentari: l'expressió a la dreta té 4 termes, cadascun representa una família de pols/zeros de $\zeta(s)\frac{x^s}{s}$: el $x$ representa el pol en $s=0$ de $\zeta$; la suma representa els zeros no trivials de $\zeta(s)$; el $\zeta'(0)/\zeta(0)$ representa el zero de $\frac{x^s}{s}$, i el logaritme representa els zeros trivials de $\zeta$.}
\pf{
Considerem el següent contorn:
\begin{center}
    \begin{tikzpicture}[domain=0:3, scale=0.5]
    \draw[ultra thick] (-5.3,0)--(3.3,0);
    \draw[ultra thick] (0,-2.5)--(0,2.5);
    \draw[thick] (3,2)--(-5,2)--(-5,-2)--(3,-2)--cycle;
    \draw[->] (3,1)--(3,1.1);
    \draw[->] (3,-1.1)--(3,-1);
    \draw[->] (-1,2)--(-1.1,2);
    \draw[<-] (-5,1)--(-5,1.1);
    \draw[<-] (-5,-1.1)--(-5,-1);
    \draw[<-] (-1,-2)--(-1.1,-2);
    \draw[dashed] (1,-2.5)--(1,2.5);
    \node at (3,2.5) {$c+iT$};
    \node at (3,-2.5) {$c-iT$};
    \node at (-5,2.5) {$-U+iT$};
    \node at (-5,-2.5) {$-U-iT$};
    \node at (-1,2.5) {$\gamma_1$};
    \node at (-1,-2.5) {$\gamma_3$};
    \node at (-5.5,0.5) {$\gamma_2$};
    \end{tikzpicture}
\end{center}
On $U$ és un senar positiu; i $T$ no és la ordenada de cap zero de $\zeta$.\\
Recordem per l'exercici 4 del FP 8, $\frac{\zeta'(s)}{\zeta(s)}$ té un pol a cada pol/zero de $\zeta$, amb residu l'ordre del zero o pol. Per tant:
\[
    \frac{1}{2\pi i}\int_{c-iT}^{c+iT}F(s)\dd s=-R
    \underbrace{-\frac{1}{2\pi i}\int_{c+iT}^{-U+iT}F(s)\dd s}_{I_1}
    \underbrace{-\frac{1}{2\pi i}\int_{-U+iT}^{-U-iT}F(s)\dd s}_{I_2}
    \underbrace{-\frac{1}{2\pi i}\int_{-U-iT}^{c-iT}F(s)\dd s}_{I_3}
\]
On $R$ són els residus:
\[
    R=\overbrace{\sum_{\substack{\rho=\beta+i\gamma\\-U<\beta<1\\|\gamma|<T}}\mathrm{Res}_{s=\rho}\,F(s)}^{R_1}+\overbrace{\mathrm{Res}_{s=0}\,F(s)}^{R_2}+\overbrace{\mathrm{Res}_{s=1}\,F(s)}^{R_3}
\]
Notem que l'ordre dels zeros de $R_1$ és 1 (tant dels zeros trivials, com el dels no trivials); i els altres residus són:
\begin{align*}
    R_1=&\sum_{\substack{\rho=\beta+i\gamma\\-U<\beta<c\\|\gamma|<T}}\frac{x^{\rho}}{\rho}+\sum_{2\leq 2m<U}\frac{x^{-2m}}{-2m}\\
    R_2=&\frac{\zeta'(0)}{\zeta(0)}\\
    R_3=&x
\end{align*}
Per tant, ens queda:
\[
    \frac{1}{2\pi i}\int_{c-iT}^{c+iT}F(s)\dd s=
-\sum_{\substack{\rho=\beta+i\gamma\\-U<\beta<c\\|\gamma|<T}}\frac{x^{\rho}}{\rho}+\sum_{2\leq 2m<U}\frac{x^{-2m}}{-2m}-\frac{\zeta'(0)}{\zeta(0)}-x
    +I_1+I_2+I_3
\]
I ara, pel lema 9:
\lema{Lema 9}{Les integrals:
\begin{align*}    
    I_1,I_3&=\bigO\left(\frac{x(\log T)^2}{T\log x}\right)\\
    I_2&=\bigO\left(\frac{T\log U}{U x^U}\right)
\end{align*}
}
Aleshores, prenent $U\to \infty$, tenim:
\[
    \frac{1}{2\pi i}\int_{c-iT}^{c+iT}F(s)\dd s=x-\sum_{\substack{\rho=\beta+i\gamma\\|\gamma|<T}}\frac{x^\rho}{\rho}-\underbrace{\frac{1}{2}\sum_{n\geq1}\frac{x^{-2m}}{-m}}_{\text{Taylor de }\log(1-x^2)}-\frac{\zeta'(0)}{\zeta(0)}+\bigO\left(\frac{x(\log T)^2}{T\log x}\right)
\]
Que és el que volíem veure.
}
\newpage
\chapter{Demostració de Lema 5 del capítol 6 donat el seu enunciat.}
\lema{Lema 5}{Sigui 
\[
    \delta(y)=\left\{\begin{array}{cl}0&\text{si }y\in(0,1)\\\frac12&\text{si }y=1\\1&\text{si }y>1\end{array}\right.
\]
Aleshores, per $y>0$, $c>0$, $T>0$:
\[
    \left|\frac{1}{2\pi i}\int_{c-iT}^{c+iT}\frac{y^s}{s}\dd s-\delta(y)\right|<\left\{\begin{array}{cl}
        \frac{y^c}{T|\log y|}&\text{si }y\neq 1\\
        \frac{c}{T}&\text{si }y=1.
    \end{array}\right.
\]
}
\pf{
Per fer aquesta demostració distingirem 3 casos: $y\in(0,1)$, $y=1$ i $y>1$.\\
\textbf{Cas 1: $y\in(0,1)$.}\\
Considerem el següent contorn:
\begin{center}
    \begin{tikzpicture}[scale = 0.5]
        \draw [ultra thick] (-0.3,0)--(5.3,0)  (0,-3.3)--(0,3.3);
        \draw (5,-3)--(5,3)--(2,3)--(2,-3)--cycle;
        \draw [->] (5,1.5)--(5,1.6);
        \draw [->] (5,-1.6)--(5,-1.5);
        \draw [->] (2,1.5)--(2,1.6);
        \draw [->] (2,-1.6)--(2,-1.5);
        \draw [->] (3.5,3)--(3.4,3);
        \draw [<-] (3.5,-3)--(3.4,-3);
        \node at (2.3,0.3) {$c$};
        \node at (2,-3.4) {$c-iT$};
        \node at (2,3.4) {$c+iT$};
        \node at (5.4,0.4) {$U$};
        \node at (3.6,-2.6) {$\gamma_1$};
        \node at (5.4,-1) {$\gamma_2$};
        \node at (3.6,2.6) {$\gamma_3$};
    \end{tikzpicture}
\end{center}
Aleshores, com que a dins d'aquest contorn, la funció $y^s/s$ no té pols, podem canviar aquesta integral per les altres tres integrals:
\[
    \frac{1}{2\pi i}\int_{c-iT}^{c+iT}\frac{y^s}{s}=
    \underbrace{\frac{1}{2\pi i}\int_{c-iT}^{U-iT}\frac{y^s}{s}}_{=I_1(U)}+
    \underbrace{\frac{1}{2\pi i}\int_{U-iT}^{U+iT}\frac{y^s}{s}}_{=I_2(U)}+
    \underbrace{\frac{1}{2\pi i}\int_{U+iT}^{c+iT}\frac{y^s}{s}}_{=I_3(U)}
\]
Però com que en $\gamma_2$ la integral és fàcil de fitar:
\[
    |I_2|\leq\int_{\gamma_2}\left|\frac{y^s}{s}\right|\dd s\leq 2T \frac{y^U}{U}\xrightarrow{U\to\infty}0
\]
Només ens hem de preocupar de $I_1=\lim_{U\to\infty}I_1(U)$ i de $I_3=\lim_{U\to\infty}I_3(U)$.
I ara per fitar les altres integrals:
\[
    |I_1|\leq\frac{1}{2\pi}\int_c^\infty\frac{|y^{\s-iT}|}{|\s-iT|}\dd \s
\]
I com que $|\s-iT|>|T|$:
\[
  \frac{1}{2\pi}\int_c^\infty\frac{|y^{\s-iT}|}{|\s-iT|}\dd \s\leq\frac{1}{2\pi}\int_c^\infty\frac{y^\s}{T}\dd \s=\frac{1}{2\pi T}\left[\frac{y^\s}{\log y}\right]_c^\infty=\frac{y^c}{2\pi T|\log y|}
\]
I si repetim aquest mateix càlcul amb l'altra integral ens dona:
\[
    |I_3|\leq\frac{y^\s}{2\pi T|\log y|}
\]
I si ho ajuntem tot, obtenim una fita millor que la que volíem.\\
\textbf{Cas 2: $y>1$.}\\
Considerem ara, aquest contorn:
\begin{center}
    \begin{tikzpicture}[scale=0.5]
        \draw [ultra thick] (-4.3,0)--(1.8,0) (0,-3.3)--(0,3.3);
        \draw (1.5,3)--(-4,3)--(-4,-3)--(1.5,-3)--cycle;
        \draw [->] (1.5,1.5)--(1.5,1.6);
        \draw [->] (1.5,-1.6)--(1.5,-1.5);
        \draw [->] (-4,1.5)--(-4,1.6);
        \draw [->] (-4,-1.6)--(-4,-1.5);
        \draw [->] (-1.25,3)--(-1.15,3);
        \draw [->] (-1.15,-3)--(-1.25,-3);
        \node at (1.8,0.3) {$c$};
        \node at (-3.6,0.4) {$-U$};
        \node at (1.5,3.4) {$c+iT$};
        \node at (1.5,-3.4) {$c-iT$};
        \node at (-4,-3.4) {$-U-iT$};
        \node at (-4,3.4) {$-U+iT$};
        \node at (-1.25,-2.6) {$\gamma_1$};
        \node at (-4.4,-1) {$\gamma_2$};
        \node at (-1.25,2.6) {$\gamma_3$};
        \node [circle, draw = black,  fill=red, inner sep=0pt,minimum size=5pt] at (0,0) {};
    \end{tikzpicture}
\end{center}
Ara no podem fer el que hem fet abans: ara en aquest contorn hi ha un pol en el $s=0$. Per tant, haurem de sumar el residu del pol:
\[
    \frac{1}{2\pi i}\int_{c-iT}^{c+iT}\frac{y^s}{s}\dd s=\mathrm{Res}_{s=0} \frac{y^s}{s}+I_1(-U)+I_2(-U)+I_3(-U)
\]
I de la mateia manera que abans, $I_2(-U)\xrightarrow{U\to\infty}0$, i les altres 2 integrals es poden fitar per:
\[
    |I_1|\leq\frac{1}{2\pi}\int_{-\infty}^c \frac{y^\s}{T}\dd \s=\frac{1}{2\pi T}\left[\frac{y^\s}{\log y}\right]_{-\infty}^c=\frac{y^c}{2\pi T |\log y|}
\]
I l'altra integral es pot fitar igual.\\
\textbf{Cas 3: $y=1$.}\\
Com que la funció és més senzilla, podem fer-ho d'una manera més directa:
\begin{align*}
    \frac{1}{2\pi i}\int_{c-iT}^{c+iT}\frac{1}{s}\dd s&=
    \frac{1}{2\pi i}\int_{-T}^{T}\frac{i}{c+it}\dd t =\\
    &=\frac{1}{2\pi}\int_{-T}^{T}\frac{c-it}{|c+it|^2}\dd t =\\
    &=\frac{1}{2\pi}\int_{-T}^{T}\frac{c-it}{c^2+t^2}\dd t =\\
    &=\frac{1}{2\pi}\int_{-T}^{T}\frac{c}{c^2+t^2}\dd t 
    -\frac{i}{2\pi}\underbrace{\int_{-T}^{T}\frac{t}{c^2+t^2}\dd t}_{=0}
\end{align*}
Aquí la segona integral és zero per la paritat de la funció que estem integrant.
\begin{align*}
    \frac{1}{2\pi}\int_{-T}^{T}\frac{c}{c^2+t^2}\dd t 
    &=\frac{1}{2\pi}\int_{-T/c}^{T/c}\frac{1}{1+u^2}\dd u =\\
    &=\frac{1}{\pi}\int_{0}^{T/c}\frac{1}{1+u^2}\dd u =\\
    &=\frac{1}{\pi}\left(\underbrace{\int_{0}^{\infty}\frac{1}{1+u^2}\dd u}_{=\frac{\pi}{2}}-\int_{T/c}^{\infty}\frac{1}{1+u^2}\dd u\right)
\end{align*}
Per tant:
\[
    \left|\frac{1}{2\pi i}\int_{c-iT}^{c+iT}\frac{1}{s}\dd s-\frac{1}{2}\right|\leq\frac{1}{\pi}\int_{T/c}^\infty\frac{1}{1+u^2}\dd u\leq \int_{T/c}^{\infty}\frac{1}{u^2}\dd u= \frac{1}{\pi}\left[\frac{-1}{u}\right]_{T/c}^\infty=\frac{c}{\pi T}<\frac{c}{T}
\]
Que és el que volíem veure.    
}
\end{document}