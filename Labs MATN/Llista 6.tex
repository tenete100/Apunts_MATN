\input{Config.tex}

\renewcommand{\thesubsection}{}
\renewcommand{\thesection}{}

\title{Mètodes analítics en teoria de nombres}
\author{Bernat Esteve Sagarra}

\begin{document}
\pagestyle{fancy}
% 
\lhead{Bernat Esteve}
\rhead{Full de problemes 6}
\maketitle

\begin{tcolorbox}[title=\section{Exercici 1}]
    Sigui $K$ un cos, $q$ un primer senar diferent de la carecterística de $K$ i $\omega$ un arrel primitiva $q$-èsima de la unitat. Donat $a\in \FF_q^\times$, definim la \textit{suma de Gauss} relativa a $a$ i $\omega$ com:
    \[
    G_\omega(a) \vcentcolon= G(a) \vcentcolon= \sum_{m\in \FF_q^\times}\left(\frac{m}{q}\right)\omega^{ma}.
    \]
    Escrivim simplement $G$ per denotar $G(1)$. Demostreu que:
    \begin{enumerate}
        \ii $G(a)=\left(\frac{a}{q}\right)G$.\\
        \ii $G=1+\sum_{m\in\FF_q^\times}\omega^{m^2}$.
    \end{enumerate}
\end{tcolorbox}
\subsection{Exercici 1.1}
Primer, recordem que pels primers senars, el símbol de Legendre es defineix
\[
\left(\frac{a}{q}\right)=\left\{\begin{array}{cl}
    -1&\text{si $a$ no és un quadrat mòdul $q$.}\\
    0&\text{si $a=0$.}\\
    1&\text{si $a$ és un quadrat mòdul $q$.}
\end{array}\right.
\]
Per tant:
\[
\left(\frac{a}{q}\right)G(1)=
\left(\frac{a}{q}\right)\sum_{m\in\FF_q^\times}\left(\frac{m}{q}\right)\omega^m=
\sum_{m\in\FF_q^\times}\left(\frac{ma}{q}\right)\omega^m=
\]
I si $a=0$, aleshores, $G(a)=\sum_{m\in\FF_q^\times}\left(\frac{m}{q}\right)=0$, ja que la meitat dels elements de $\FF_q^\times$ són quadrats i la altra no (es pot veure veient que el símbol de Legendre és multiplicatiu, i que per tant, hi ha una bijecció entre els 2 conjunts que consisteix en multiplicar per un no-quadrat).\\
En el cas de que $a\neq 0$, aleshores $\left(\frac{ma^2}{q}\right)=\left(\frac{m}{q}\right)$. Per tant:
\[
\sum_{m\in\FF_q^\times}\left(\frac{ma}{q}\right)\omega^m=
\sum_{am\in\FF_q^\times}\left(\frac{ma^2}{q}\right)\omega^{am}=
\sum_{m\in\FF_q^\times}\left(\frac{m}{q}\right)\omega^{am}=G(a)
\]
Que és el que volíem veure.\par
\vspace{10mm}
\subsection{Exercici 1.2}
Per veure la segona part, notem que 
\[
\sum_{m\in\FF_q^\times}\left(\frac{m}{q}\right)\omega^m
\]
Però, notem que si $m=n^2$, aleshores, $\left(\frac{m}{q}\right)\omega^m=\omega^{n^2}$; d'altra banda, si $m$ no és un quadrat, aleshores 
\[
\sum_{\left(\frac{m}{q}\right)=-1}\left(\frac{m}{q}\right)\omega^m=
\sum_{\left(\frac{m}{q}\right)=-1}-\omega^m=
\sum_{\left(\frac{m}{q}\right)=1}\omega^m
\]
On la última igualtat es deu a que la suma de les arrels és 0. Per tant
\[
    \sum_{m\in\FF_q^\times}\left(\frac{m}{q}\right)\omega^m=
    2\sum_{\left(\frac{m}{q}\right)=1}\omega^m
\]
Però notem que estem sumant 2 vegades sobre tots els quadrats, però com que cada quadrat és el quadrat de 2 enters mòdul $q$. Excepte que el 1 només el sumem una vegada (estem excluint el $1$, ja que no és una arrel primitiva). Per tant:
\[
    \sum_{m\in\FF_q^\times}\left(\frac{m}{q}\right)\omega^m=
    2\sum_{\left(\frac{m}{q}\right)=1}\omega^m=
    \sum_{m\in\FF_q^\times}\omega^{m^2}
\]
Que és el que volíem veure.
\newpage
\begin{tcolorbox}[title=\section{Exercici 2}]
Sigui $K$ un cos, $q$ un primer senar diferent de la carecterística de $K$ i $\omega$ una arrel primitiva $q$-èsima de la unitat. Demostreu que:
\begin{enumerate}
    \ii Les arrels $q$-èsimes primitives de la unitat a la clausura algebràica $\ol K$ són les arrels de $\Phi_q(T)=T^{q-1}+\dots+T+1\in K[T]$. En particular, la suma de totes elles és -1.\\
    \ii $G^2=(-1)^{\frac{q-1}{2}}q.$
\end{enumerate}
\end{tcolorbox}
\subsection{Exercici 2.1}
Notem que $\Phi_q(T)=\frac{T^q-1}{T-1}$, per tant,  $\Phi_q(\omega)=\frac{1-1}{\omega-1}=0$. que és el que volia veure. I per les fòrmules de Vieta, tenim que la suma de les arrels és menys el coeficient de grau $[T^{q-2}]$, i per tant, la suma de les arrels és -1.
\vspace{10mm}
\subsection{Exercici 2.2}
Per veure què val $G^2$, seguirem la indicació:
\begin{align*}
    G^2=&\sum_{m\in\FF_q^*}\sum_{n\in\FF_q^*}\left(\frac{m}{q}\right)\left(\frac{n}{q}\right)\omega^{m}\omega^{n}=\\
    =&\sum_{m\in\FF_q^*}\sum_{n\in\FF_q^*}\left(\frac{mn}{q}\right)\omega^{m+n}=\\
    =&\sum_{m\in\FF_q^*}\sum_{n\in\FF_q^*}\left(\frac{m}{q}\right)\omega^{n(m+1)}=\\
    =&\sum_{m\in\FF_q^*}\left(\frac{m}{q}\right)\sum_{n\in\FF_q^*}\omega^{n(m+1)}
\end{align*}
Però, sabem que si $m+1\equiv 0\pmod q$ aleshores, la suma dona $q-1$, i dona $-1$ altrament, per tant:
\[
\sum_{m\in\FF_q^*}\left(\frac{m}{q}\right)\sum_{n\in\FF_q^*}\omega^{n(m+1)}=
\left(\frac{-1}{q}\right)(q-1)-\sum_{m\in\FF_q^*\setminus\{-1\}}\left(\frac{m}{q}\right)
\]
Però sabem també, que la suma sobre $\sum_{m\in\FF_q^*}\left(\frac{m}{q}\right)=0$, per tant, si restem als 2 costats $\left(\frac{-1}{q}\right)$ obtenim:
\[
    \left(\frac{-1}{q}\right)(q-1)+\left(\frac{-1}{q}\right)=
    \left(\frac{-1}{q}\right)q=(-1)^{\frac{q-1}{2}}q
\]
On la última igualtat, és un resultat bastant conegut (i.e. que $-1$ és un quadrat mòdul $q$ si i només si $q$ és congruent amb $1$ mòdul 4).
\newpage
\begin{tcolorbox}[title=\section{Exercici 3}]
    Siguin $q$ i $p$ primers senars diferents. Demostreu que
    \[
    \left(\frac{p}{q}\right)\left(\frac{q}{p}\right)=(-1)^{\frac{p-1}{2}\frac{q-1}{2}}
    \]
\end{tcolorbox}
\subsection{Exercici 3}
Sigui $\omega$ una arrel primitiva $q$-èsima. Aleshores mirem-nos $G_\omega$ sobre el cos $\FF_p[\omega]$.
\[
\left(\frac{p}{q}\right)G=G_\omega(p)=\sum_{m\in\FF_q^*}\left(\frac{m}{q}\right)\omega^{pm}
\]
però com que $p$ és un primer senar, aleshores $\left(\frac{m}{q}\right)^p=\left(\frac{m}{q}\right)$. Per tant:
\[
\sum_{m\in\FF_q^*}\left(\frac{m}{q}\right)\omega^{pm}=
\sum_{m\in\FF_q^*}\left(\left(\frac{m}{q}\right)\omega^{m}\right)^p
\]
I com que el cos té carecterística $p$, aleshores, $a^p+b^p=(a+b)^p$:
\[
\sum_{m\in\FF_q^*}\left(\left(\frac{m}{q}\right)\omega^{m}\right)^p=
\left(
\sum_{m\in\FF_q^*}\left(\frac{m}{q}\right)\omega^{m}\right)^p=G^p
\]
I com que $p$ i $q$ són primers senars, alehsores $p-1$ és parell.
\[
\left(\frac{p}{q}\right)=G^{p-1}=(-1)^{\frac{q-1}{2}\frac{p-1}{2}}
\]
Que és el que volíem veure.
\newpage
\begin{tcolorbox}[title=\section{Exercici 4}]
    Sigui $G\in\CC$ la suma de Gauss relativa a $\omega=e^{2\pi i/q}$. Demostreu que 
    \[
    G=\sqrt{q}\frac{1+i^{-q}}{1-i}.
    \]
\end{tcolorbox}
\subsection{Exercici 4}
Recordem la fòrmula de sumació de Poisson: Si $A$ i $B$ són enters amb $A<B$, aleshores
\[
\sum_{n=A}^{B}f(n)=\sum_{\nu\in\ZZ}\int_A^B f(u)e^{2\pi i \nu u}\dd u
\]
Per tant, aplicant això al nostre cas:
\[
\sum_{n=0}^{q}\omega^{n^2}=\sum_{\nu\in\ZZ}\int_0^q e^{2\pi i \nu u+\frac{2\pi iu^2}{q}}\dd u
\]
I ara si fem el canvi de variable $u=q(y-\nu/2)$ $\dd u =q\dd y$.
\[
\sum_{\nu\in\ZZ}\int_0^q e^{2\pi i \nu u+\frac{2\pi iu^2}{q}}\dd y=
\sum_{\nu\in\ZZ}\int_{\nu/2}^{1+\nu/2} e^{2\pi i \nu q(y-\nu/2)+2\pi iq(y^2-\nu y+\nu^2/4)}q\dd y
\]
I simplificant les coses, obtenim
\[
\sum_{\nu\in\ZZ}\int_{\nu/2}^{1+\nu/2} e^{2\pi i q(-\nu^2/4+y^2)}q\dd y=
q\sum_{\nu\in\ZZ}e^{-\pi i q\nu^2/2}\int_{\nu/2}^{1+\nu/2} e^{2\pi i qy^2}\dd y
\]
Si ara separem la suma entre $\nu$ parell i senar:
\[
q\sum_{2\nu\in\ZZ}e^{-2\pi i q\nu^2}\int_{\nu}^{1+\nu} e^{2\pi i qy^2}\dd y+
q\sum_{2\nu+1\in\ZZ}e^{2\pi i q\nu(-\nu^-1)}e^{-\pi i q/2}\int_{\nu+1/2}^{\nu+3/2} e^{2\pi i qy^2}\dd y
\]
Però notem que com que $\nu$ són enters, aleshores per Euler $e^{2\pi i k}=1$:
\[
q\sum_{2\nu\in\ZZ}\int_{\nu}^{1+\nu} e^{2\pi i qy^2}\dd y+
q\sum_{2\nu+1\in\ZZ}e^{-\frac{\pi i q}{2}}\int_{\nu+1/2}^{\nu+3/2} e^{2\pi i qy^2}\dd y
\]
I si ajuntem tots els sumatoris:
\[
q\int_{-\infty}^{\infty} e^{2\pi i qy^2}\dd y+qe^{-\frac{\pi i q}{2}}\int_{-\infty}^{\infty} e^{2\pi i qy^2}\dd y=
q(1+e^{-\frac{\pi i q}{2}})\int_{-\infty}^{\infty} e^{2\pi i qy^2}\dd y
\]
Però $e^{-\frac{\pi iq}{2}}=i^{-q}$. Per tant:
\[
q(1+i^{-q})\int_{-\infty}^{\infty} e^{2\pi i qy^2}\dd y
\]
I ara si fem el canvi de variable dins de la integral: $y=\frac{x}{\sqrt{q}}$:
\[
q(1+i^{-q})\int_{-\infty}^{\infty}\frac{1}{\sqrt{q}} e^{2\pi i x^2}\dd x=
\sqrt{q}(1+i^{-q})\underbrace{\int_{-\infty}^{\infty} e^{2\pi i x^2}\dd x}_{=C}=
\sqrt{q}(1+i^{-q})C
\]
Per determinar el valor de la constant, podem $q=3$:
\[
G_{\zeta_3}(1)=
\left(\frac{1}{3}\right)\zeta_3+
\left(\frac{2}{3}\right)\zeta_3^2
\]
On $\zeta_3=\cos\left(\frac{2\pi}{3}\right)+i\sin\left(\frac{2\pi}{3}\right)$, i $\zeta_3^2=\ol\zeta_3$. Per tant
\[
G_{\zeta_3}(1)=\zeta_3-\zeta_3^2=i\sqrt{3}
\]
I segons la fòrmula
\[
G=\sqrt{3}(1+i)C=i\sqrt{3}
\]
Per tant
\[
C=\frac{i}{1+i}=\frac{1}{1-i}
\]
I ens dona justament la fòrmula que havíem de veure.
\end{document}
% ---
% \newpage
% \begin{tcolorbox}[title=\section{Exercici XXX}]
% \end{tcolorbox}
% \subsection{Exercici XXX}
