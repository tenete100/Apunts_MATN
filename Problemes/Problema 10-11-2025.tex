\documentclass[11pt]{article}
\usepackage{amsmath,amssymb,amsthm}
\usepackage[catalan]{babel}
\usepackage[draft]{graphicx}
% \usepackage[usenames,dvipsnames,svgnames]{xcolor}
\usepackage{mathrsfs}
\usepackage[shortlabels]{enumitem}
\usepackage{mathtools}
\usepackage{microtype}
\usepackage{hyperref}
\usepackage{tikz}


%% Macros
\providecommand{\ol}{\overline}
\providecommand{\ul}{\underline}
\providecommand{\wt}{\widetilde}
\providecommand{\wh}{\widehat}
\providecommand{\eps}{\varepsilon}
\providecommand{\half}{\frac{1}{2}}
\providecommand{\inv}{^{-1}}
\newcommand{\dang}{\measuredangle} %% Directed angle
\providecommand{\CC}{\mathbb C}
\providecommand{\FF}{\mathbb F}
\providecommand{\NN}{\mathbb N}
\providecommand{\QQ}{\mathbb Q}
\providecommand{\RR}{\mathbb R}
\providecommand{\ZZ}{\mathbb Z}
\providecommand{\ts}{\textsuperscript}
\providecommand{\dg}{^\circ}
\providecommand{\ii}{\item}
\DeclareMathOperator*{\lcm}{lcm}
\DeclareMathOperator*{\argmin}{arg min}
\DeclareMathOperator*{\argmax}{arg max}

% theorem environments
% starred versions are not numbered, unstarred versions have a number
\theoremstyle{definition}
\newtheorem{theorem}{Theorem}
\newtheorem{lemma}[theorem]{Lemma}
\newtheorem{claim}[theorem]{Claim}
\newtheorem*{theorem*}{Theorem}
\newtheorem*{lemma*}{Lemma}
\newtheorem*{claim*}{Claim}
\theoremstyle{remark}
\newtheorem{remark}[theorem]{Remark}
\newtheorem*{remark*}{Remark}

\begin{document}

%% Insert problem statement here
\paragraph{TSTST 2018}
Per un enter $n>0$ definim $\mathcal{F}(n)$ com el conjunt de $m\in\ZZ_{\geq1}$ tal que el polinomi
\[
p(x)=x^2+mx+n
\]
té alguna arrel entera.
\begin{enumerate}
    \ii sigui $S$ el conjunt de $n$ tals que $\mathcal{F}(n)$ conté 2 números consecutius. Aleshores, demostra que $S$ és infinit i que tenim:
    \[
    \sum_{n\in S} \frac{1}{n}\leq 1.
    \]
\end{enumerate}
\newpage
\paragraph{Pistes:}
\begin{enumerate}
    \ii Els $n\in S$ menors que $100$, són: $4$, $12$, $24$, $36$, $40$, $60$, $72$.
    \vspace{45mm}
    \ii Què diuen les fòrmules de Vieta?\\
    \vspace{45mm}
    \ii Pots calcular $\mathcal{F}(n)$ donat que $n=60$, per exemple?\\

\end{enumerate}
\newpage

\paragraph{Solució:}
Notem què diuen les fòrmules de Vieta: si $a$ i $b$ són arrels de $p(x)=x^2+mx+n$, aleshores tenim:
\[
    p(x)=(x-x_0)(x-x_1)=x^2-(x_1+x_1)x+x_0x_1 \qquad \implies \qquad x_0+x_1=-m \text{ i }x_0x_1=n.
\]
Per tant, sabem que si $n\in S$, aleshores, existeixen dos $d_1$, $d_2\in\ZZ_{\geq1}$ divisors de $n$ tals que 
\[
    m=\underbrace{\frac{n}{d_1}}_{k_1}+d_1 \qquad m+1=\underbrace{\frac{n}{d_2}}_{k_2}+d_2
\]
són consecutius. Per tant, podem escriure:
\[
    n=abcd
\]
On $ab=k_1$, $cd=d_1$; i $ac=k_2$, $bd=d_2$. Si fem aquest canvi de variables:
\begin{align*}
    ab+cd&=m+1\\
    ac+bd&=m
\end{align*}
Per tant:
\[
    1=ab+cd-ac-bd=a(b-c)-d(b-c)=(a-d)(b-c)
\]
Però com que estem sobre els enters, la única manera de multiplicar 2 números i que doni 1 ha de ser amb $(+1)(+1)$ o $(-1)(-1)$. Per tant, si $a-d=1=b-c$, aleshores aconseguim que $n=(c^2+c)(d^2+d)$. I si fem $a-d=-1=b-c$ ens dona $n=(a^2+a)(b^2+b)$, que és el mateix que teníem abans.\par
\vspace{15mm}
Ara per veure quant val el sumatori:
\[
    \sum_{n\in S}\frac{1}{n}\leq
    \sum_{a\geq1}\sum_{b\geq1}\frac{1}{(a+1)a(b+1)b}=
    \sum_{a\geq1}\frac{1}{(a+1)a}\sum_{b\geq1}\frac{1}{(b+1)b}
\]
Però notem que 
\[
    \sum_{n\geq1}\frac{1}{n^2+n}=\sum_{n\geq1}\frac{1}{n}-\frac{1}{n+1}=1
\]
Que ens dona que el sumatori és exactament 1; que és el que volíem veure.
\end{document}