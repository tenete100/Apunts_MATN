\documentclass[11pt]{article}
\usepackage{amsmath,amssymb,amsthm}
\usepackage[catalan]{babel}
\usepackage[draft]{graphicx}
% \usepackage[usenames,dvipsnames,svgnames]{xcolor}
\usepackage{mathrsfs}
\usepackage[shortlabels]{enumitem}
\usepackage{mathtools}
\usepackage{microtype}
\usepackage{hyperref}
\usepackage{tikz}


%% Macros
\providecommand{\ol}{\overline}
\providecommand{\ul}{\underline}
\providecommand{\wt}{\widetilde}
\providecommand{\wh}{\widehat}
\providecommand{\eps}{\varepsilon}
\providecommand{\half}{\frac{1}{2}}
\providecommand{\inv}{^{-1}}
\newcommand{\dang}{\measuredangle} %% Directed angle
\providecommand{\CC}{\mathbb C}
\providecommand{\FF}{\mathbb F}
\providecommand{\NN}{\mathbb N}
\providecommand{\QQ}{\mathbb Q}
\providecommand{\RR}{\mathbb R}
\providecommand{\ZZ}{\mathbb Z}
\providecommand{\ts}{\textsuperscript}
\providecommand{\dg}{^\circ}
\providecommand{\ii}{\item}
\DeclareMathOperator*{\lcm}{lcm}
\DeclareMathOperator*{\argmin}{arg min}
\DeclareMathOperator*{\argmax}{arg max}

% theorem environments
% starred versions are not numbered, unstarred versions have a number
\theoremstyle{definition}
\newtheorem{theorem}{Theorem}
\newtheorem{lemma}[theorem]{Lemma}
\newtheorem{claim}[theorem]{Claim}
\newtheorem*{theorem*}{Theorem}
\newtheorem*{lemma*}{Lemma}
\newtheorem*{claim*}{Claim}
\theoremstyle{remark}
\newtheorem{remark}[theorem]{Remark}
\newtheorem*{remark*}{Remark}

\begin{document}

%% Insert problem statement here
\paragraph{TSTST 2018}
Per un enter $n>0$ definim $\mathcal{F}(n)$ com el conjunt de $m\in\ZZ_{\geq1}$ tal que el polinomi
\[
p(x)=x^2+mx+n
\]
té alguna arrel entera.
\vspace{5mm}\\
La setmana passada vam veure que el conjunt $S$ de $n$ tals que $\mathcal{F}(n)$ conté 2 números consecutius és infinit i la suma dels recíprocs dels elements de $S$ és menor que 1.\\
\begin{enumerate}
    \ii Demostra que hi han infinits $n$ tals que $F(n)$ conté tres enters consecutius.
\end{enumerate}
\newpage
Pistes:
\newpage
\paragraph{Solució:}
Notem que si $\mathcal{F}(n)$ té tres elements consecutius, aleshores pel mateix raonament de la setmana passada, tenim que existeixen $d_0$, $d_1$ i $d_2$ tals que
\[
    m=\underbrace{\frac{n}{d_0}}_{k_0}+d_0 \qquad
    m+1=\underbrace{\frac{n}{d_1}}_{k_1}+d_1 \qquad
    m+2=\underbrace{\frac{n}{d_2}}_{k_2}+d_2
\]
Per tant, pel que hem trobat abans:
\[
    n=ab(b+1)(a+1)=a'b'(b'+1)(a'+1)
\]
tal que
\[
    \left.\begin{array}{l}
    2ab+a+b+1=(a+1)(b+1)+ab=m+2\\
    2ab+a+b=m+1\\
    2a'b'+a'+b'+1=(a'+1)(b'+1)+a'b'=m+1\\
    2a'b'+a'+b'=m
    \end{array}\qquad\right|\qquad
    \begin{array}{l}
    2ab+a+b+1=m+2\\
    2a'b'+a'+b'=m
    \end{array}\qquad
\]

\[
    \begin{array}{ll}
        ((a+1)(b+1)-(a'+1)(b'+1))+ab-a'b'-1\\
        ab(a+1)(b+1)-a'b'(a'+1)(b'+1)=0\\
        2(ab-a'b')+(a-a')+(b-b')=1\\
        \frac{2a'b'+a'+b'+1}{2ab+a+b}=1
    \end{array}
\]

\[
    2ab+a+b=(a+1)(b+1)+ab-1
\]
%}

\end{document}
% d_2n + d_1^2d_2 + d_1d_2 - d_1n - d_2^2d_1 =\\
% d_1d_2(d_1 - d_2) + d_1d_2 + n(d_2 - d_1)


( 1 )( 1 )( 2 )( 2 ) 4
( 1 )( 2 )( 2 )( 3 ) 12
( 1 )( 3 )( 2 )( 4 ) 24
( 1 )( 4 )( 2 )( 5 ) 40
( 1 )( 5 )( 2 )( 6 ) 60
( 1 )( 6 )( 2 )( 7 ) 84
( 1 )( 7 )( 2 )( 8 ) 112
((1 )( 8 )( 2 )( 9 ) 144)
((1 )( 9 )( 2 )(10 ) 180)
( 1 )(10 )( 2 )(11 ) 220
( 1 )(11 )( 2 )(12 ) 264
( 1 )(12 )( 2 )(13 ) 312
( 1 )(13 )( 2 )(14 ) 364
( 1 )(14 )( 2 )(15 ) 420
( 1 )(15 )( 2 )(16 ) 480
( 1 )(16 )( 2 )(17 ) 544
( 1 )(17 )( 2 )(18 ) 612
( 1 )(18 )( 2 )(19 ) 684
( 1 )(19 )( 2 )(20 ) 760
(( 1 )(20 )( 2 )(21 ) 840)
( 1 )(21 )( 2 )(22 ) 924
( 1 )(22 )( 2 )(23 ) 1012*

( 2 )( 2 )( 3 )( 3 ) 36
( 2 )( 3 )( 3 )( 4 ) 72
( 2 )( 4 )( 3 )( 5 ) 120
((2 )( 5 )( 3 )( 6 ) 180)
( 2 )( 6 )( 3 )( 7 ) 252
( 2 )( 7 )( 3 )( 8 ) 336
( 2 )( 8 )( 3 )( 9 ) 422
( 2 )( 9 )( 3 )(10 ) 540
( 2 )(10 )( 3 )(11 ) 660
( 2 )(11 )( 3 )(12 ) 792
( 2 )(12 )( 3 )(13 ) 936
( 2 )(13 )( 3 )(14 ) 1092
(( 2 )(14 )( 3 )(15 ) 1260)*

(( 3 )( 3 )( 4 )( 4 ) 144)
( 3 )( 4 )( 4 )( 5 ) 240
( 3 )( 5 )( 4 )( 6 ) 360
( 3 )( 6 )( 4 )( 7 ) 504
( 3 )( 7 )( 4 )( 8 ) 672
( 3 )( 8 )( 4 )( 9 ) 864
( 3 )( 9 )( 4 )(10 ) 1080*

( 4 )( 4 )( 5 )( 5 ) 400
( 4 )( 5 )( 5 )( 6 ) 600
(( 4 )( 6 )( 5 )( 7 ) 840)
( 4 )( 7 )( 5 )( 8 ) 1120
( 4 )( 8 )( 5 )( 9 ) 1440

( 5 )( 5 )( 6 )( 6 ) 900
(( 5 )( 6 )( 6 )( 7 ) 1260)*


4,12,24,36,40,60,72,84,112,120,144,144,180,180,220,240,264,252,312,336,360,364,400,420,422,480,504,540,544,600,612,660,672,684,760,792,840,840,864,924,900,936,1012,1080,1092,1120,1260,1260,1440




144:
2*2*2*2*3*3
1,144   145     2,72    74      3,48    51      4,36    40      6,24    30
8,18    26      (4*2)(2*3*3)
9,16    25      (4*2*2)(3*3)
12,12   24      (4*3)(2*2*3)

180:
2*2*3*3*5
1,180   181     2,90    92      3,60    63      4,45    49      5,36    41
6,30    36
9,20    29      (3*3)(2*2*5)
10,18   28      (2*5)(2*3*3)
12,15   27      (2*2*3)(3*5)

840: NO
2*3*4*5*7
1,840   841     2,420   422     3,280   283     4,210   214     5,168   173
6,140   146     7,120   127     8,105   113     10,84   94      12,70   82
14,60   74      15,56   71
(20,42   62)    (4*5)(2*3*7)
(21,40   61)    (4*2*5)(3*7)
(24,35   59)    (4*2*3)(5*7)
(28,30   58)    (4*7)(2*3*5)

1260
2*2*3*3*5*7
1,1260  1261    2,630   632     3,420   423     4,315   319     5,252   257
6,210   216     7,180   187     9,140   149     10,126  136     12,105  127
14,90   104     15,84   99      18,70   88      20,63   83      21,60   81
28,45   73      (2*2*7)(3*3*5)
30,42   72      (2*3*5)(2*3*7)
35,36   71      (2*2*3*3)(5*7)