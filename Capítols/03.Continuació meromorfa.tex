\chapter{Continuació meromorfa i equació funcional de $\zeta(s)$}
Al 1859, Riemann va publicar un article on introduïa la funció zeta de Riemann i on establia dues propietats fonamentals d'aquesta funció: la seva continuació meromorfa a tot el pla complex i l'equació funcional que satisfà. En aquest capítol veurem com demostrar aquestes propietats. \par
Primer definirem la funció $theta$ de Jacobi, i la funció $\Gamma$.
\dfe{Funció $\theta$ de Jacobi}{
    La funció $\theta$ de Jacobi es defineix $\theta\colon\RR_{>0}\to\RR_{>0}$ com:
    \[
    \theta(x)=\sum_{n\in\ZZ}e^{-n^2\pi x}
    \]
    I defnim també, la funció $\omega$ com:
    \[
    \omega(x)=\sum_{n=\in\ZZ_{\geq1}} e^{-n^2\pi x}=\frac{\theta(x)-1}{2}
    \]
}
I ara hem de veure que aquestes funcions estan ben definides, és a dir, que les sèries convergeixen; i com es comporten per $x>1$.
\lema{}{\label{4.01_lema}
\begin{enumerate}
    \ii La funció $\omega(x)$ és convergent per tot $x>0$.\\
    \ii El comportament de $\omega(x)$ per $x>1$ és:
    \[
    \omega(x)=\mathcal{O}(e^{-\pi x})
    \]
    \ii El mateix val per la funció $\theta(x)$.
\end{enumerate}
}
\pf{
\[
\omega(x)=\sum_{n\geq1}e^{-n^2\pi x}\leq\sum_{n\geq1}e^{-n\pi x}=\frac{e^{-\pi x}}{1-e^{-\pi x}}
\]
On la última igualtat és certa perquè per $x>0$ tenim que $e^{-\pi x}<1$ i per tant la sèrie geomètrica convergeix. Així doncs, la sèrie de $\omega(x)$ convergeix per tot $x>0$ i a més, per $x>1$ tenim que:
\[
\frac{e^{-\pi x}}{1-e^{-\pi x}}=\mathcal{O}(e^{-\pi x})
\]
Ja que $1-e^{-\pi x}\neq 0$, i per tant es pot fitar per una constant.\par
Finalment, com que $\theta(x)=1+2\omega(x)$, el mateix val per $\theta(x)$.
}
Ara veurem la equació funcional de la funció $\theta$.
\teorema{equació funcional de $\theta(x)$}{\label{4.02_teorema equació funcional de theta} La funció $\theta(x)$ de Jacobi satisfà la següent equació funcional:
\[
\theta\left(\frac{1}{x}\right)=\sqrt{x}\,\theta(x)
\]
}
Per fer aquesta demostració necessitarem resultats de l'assignatura d'anàlisi matemàtica. Però abans, afegim una mica de notació.\par
Sigui $f\colon \RR\to \RR$ una funció contínua i monòtona a trossos. Descrivim
\[
f_1(u)=\left\{\begin{array}{ll}
    f(\{u\}) & \text{si } u\not \in \ZZ\\
    \frac{f(0)+f(1)}{2}& \text{si } u\in \ZZ.
\end{array}\right.
\]
I ara amb aquesta funció $f_1$, que és una funció periòdica de període 1, defnim els seus coeficients de Fourier:
\teorema{Expansió de Fourier.}{\label{4.13_teorema resultat d'anmat}
Sigui $f_1$ una funció com la que acabem de veure, aleshores, la podem expressar de la següent manera:
\[
f_1(u)=\frac{a_0}{2}+\sum_{\nu\geq1}(a_{\nu}\cos(2\pi \nu u)+b_\nu\sin(2\pi \nu u))
\]
On els coeficients (anomenats coeficients de Fourier) venen donats per les següents fórmules:
\[
\frac{a_\nu}{2}=\int_0^1f(t)\cos(2\pi \nu t)dt\quad\text{i}\quad\frac{b_\nu}{2}=\int_0^1f(t)\sin(2\pi \nu t)dt
\]
}
Aquest teorema, degut a que és temari d'una altra assignatura, no el demostrarem. Ara donarem un altre teorema que ens serà útil per a la demostració de l'equació funcional de $\theta$.
\teorema{Fórmula de sumació de Poisson}{\label{4.14_teorema Poisson summation} Sigui $f$ una funció com abans. Sigui $A$, $B\in \ZZ$, amb $A<B$. Aleshores:
\[
\sideset{}{'}\sum_{n=A}^B f(n)=\sum_{\nu\in\ZZ}\int_a^Bf(t)e^{2\pi i\nu t}\dd t
\]
On $\sideset{}{'}\sum$ denota que s'ha de fer la mitjana en els extrems.


}
