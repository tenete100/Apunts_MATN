\documentclass{report}
% Configuración
%----------------
%   Importaciones
%----------------
\usepackage{xparse,xcolor}
\usepackage{sectsty}
\usepackage{amsmath,amssymb,latexsym,cancel,amsthm,mathtools} %amsfonts
\usepackage{tikz,pgfplots}
\pgfplotsset{compat=1.18, width=10cm}
\usepackage{multicol}
\usepackage{tikz-cd}
\usepackage{enumitem}

\usepackage[colorlinks=true]{hyperref}
\usepackage[most]{tcolorbox}
\usepackage{blindtext}
\usepackage{framed}
\usepackage{titletoc}
\usepackage{etoolbox}

\usepackage[explicit]{titlesec}
\usepackage{anyfontsize}

\usepackage[a4paper]{geometry}
\usepackage{graphicx, wrapfig, subcaption, setspace, booktabs}
\usepackage[T1]{fontenc}
\usepackage[catalan]{babel}
\usepackage[scaled]{helvet}     % Fuente del documento


\renewcommand{\familydefault}{\rmdefault} % per tornar a canviar la font posar aquí \sfdefault
\usepackage[utf8]{inputenc}
\usepackage{url, lipsum}
\usepackage{tabularx}
\usepackage{cancel}

% \setlength{\parindent}{0cm}
% \setlength{\parskip}{5pt}

%-----------------
%   Colores
%-----------------
\definecolor{mytheorembg}{HTML}{F2F2F9}
\definecolor{mylenmabg}{HTML}{FFFAF8}
\definecolor{mylenmafr}{HTML}{983b0f}
\definecolor{mypropbg}{HTML}{f2fbfc}
\definecolor{mypropfr}{HTML}{191971}
\definecolor{myp}{RGB}{197, 92, 212}
\definecolor{primary}{HTML}{207ba5}    % Color principal
\definecolor{migris}{RGB}{17, 17, 17}
\definecolor{grisfondo}{RGB}{249, 249, 249}
\definecolor{MyGrey}{HTML}{5B5B5B}


%----------------
%   Colores para
%   Las urls
%----------------
\hypersetup{
    colorlinks=true,
    linkcolor=black,
    filecolor=magenta,
    urlcolor=blue,
}

%----------------
%   Cajas
%----------------


% \newcommand{\comb}[2]{\begin{pmatrix}
%      #1\\
%      #2
% \end{pmatrix}}

\newcommand\fancybox[3]{%
    \tcbset{
        mybox/.style={
                enhanced,
                boxsep=0mm,
                opacityfill=0,
                overlay={
                        \coordinate (X) at ([xshift=-1mm, yshift=-1.5mm]frame.north west);
                        \node[align=right, text=#1, text width=2.5cm, anchor=north east] at (X) {\bf#2};
                        \draw[line width=0.5mm, color=#1] (frame.north west) -- (frame.south west);
                    }
            }
    }
    \begin{tcolorbox}[mybox]
        #3
    \end{tcolorbox}
}

\tcbuselibrary{theorems,skins,hooks}
\NewDocumentCommand\caja{m O{\Large #1} O{grisfondo} O{primary} O{number within=chapter}}
{
    \newtcbtheorem[#5]{#1}{\large #2}
    {%
        enhanced
        ,breakable
        ,colback = #3
        ,frame hidden
        ,boxrule = 0sp
        ,borderline west = {2pt}{0pt}{#4}
        ,sharp corners
        ,detach title
        ,before upper = \tcbtitle\par\smallskip
        ,coltitle = #4
        ,fonttitle = \bfseries%
        % ,description font = \mdseries
        ,separator sign none
        ,segmentation style={solid, #4}
    }
    {th}
}

\caja{Corolari}[Corol·lari][myp!10][myp!85!black]
\caja{Lema}[Lema][mylenmabg][mylenmafr]
\caja{Propo}[Proposició][mypropbg][mypropfr]
\caja{defi}[Definició][primary!12][primary]
\caja{Teorema}[Teorema][primary!12][primary]
\caja{Nota}[Nota][white][migris][no counter]
\caja{Propietats}[Propietats][white][migris][no counter]
\caja{preg}[Pregunta][white][migris][no counter]
\caja{Claim}[Claim:][white][migris][no counter]

%---------------
%   Comandos
%---------------
\newcommand{\teorema}[2]{\begin{Teorema}{#1}{}#2\end{Teorema}}
\newcommand{\pregunta}[2]{\begin{preg}{#1}{}#2\end{preg}}
\newcommand{\coro}[2]{\begin{Corolari}{#1}{}#2\end{Corolari}}
\newcommand{\lema}[2]{\begin{Lema}{#1}{}#2\end{Lema}}
\newcommand{\Prop}[2]{\begin{Propo}{#1}{}#2\end{Propo}}
\newcommand{\propietats}[2]{\begin{Propietats}{#1}{}#2\end{Propietats}}
\newcommand{\nota}[2]{\begin{Nota}{#1}{}{\em\color{MyGrey}#2}\end{Nota}}
\newcommand{\dfe}[2]{\begin{defi}{#1}{}#2\end{defi}}
\newcommand{\pf}[1]{\begin{proof}[\color{primary}\textbf{Demostració.}] #1 \end{proof}\vspace{7mm}}
\newcommand{\claim}[1]{\begin{Claim}{}{} #1 \end{Claim}}

\theoremstyle{definition}
\newtheorem*{exem}{\color{primary}Exemple}
\newcommand{\exemple}[1]{\begin{exem}#1\end{exem}}

\theoremstyle{definition}
\newtheorem*{solu}{\color{primary}Solució}
\newcommand{\sol}[1]{\begin{solu}#1\end{solu}}

\theoremstyle{definition}
\newtheorem*{obser}{\color{primary}Observació}
\newcommand{\obs}[1]{\begin{obser}#1\end{obser}}

%---------------
%   Listas
%---------------

% \newcommand{\cnumero}[2]{
%     \tikz[baseline=(myanchor.base)]
%     \node[minimum size=0.2cm,circle,
%         inner sep=1pt,draw, #2,thick,fill=#2](myanchor)
%     {\color{white}\bfseries\fontsize{8}{8}#1};}

\newcommand*{\itembolasazules}[1]{\protect\cnumero{#1}{primary}}
    
\newcommand{\listo}[1]{
    \begin{enumerate}[label=\itembolasazules{\arabic*}]
        #1
    \end{enumerate}
}

\newcommand{\listu}[1]{
    \begin{itemize}[label=$\color{primary} \bullet$]
        #1
    \end{itemize}
}

%-------------------------
% Tabla de Contenidos
%-------------------------

\patchcmd{\tableofcontents}{\contentsname}{\contentsname}{}{}

\renewenvironment{leftbar}
{\def\FrameCommand{\hspace{6em}%
        {\color{primary}\vrule width 2pt depth 6pt}\hspace{1em}}%
    \MakeFramed{\parshape 1 0cm \dimexpr\textwidth-6em\relax\FrameRestore}\vskip2pt%
}
{\endMakeFramed}

\titlecontents{chapter}[0em]
{\vspace*{2\baselineskip}}
{\parbox{4.5em}{%
        \hfill\Huge\bfseries\color{primary}\thecontentslabel}
    \vspace*{-2.3\baselineskip}\leftbar\textbf{\color{primary}\small\chaptername~\thecontentslabel}\\
}{}{\endleftbar}

\titlecontents{section}[8.4em]
{\contentslabel{3em}}{}{} 
{\hspace{0.5em}\nobreak\itshape\color{primary}\contentspage}

\titlecontents{subsection}[8.4em]
{\contentslabel{3em}}{}{} 
{\hspace{0.5em}\nobreak\itshape\color{primary}\contentspage}

%-----------------------------
%   Formato de los capitulos
%-----------------------------

%==================
% Capitulos
%==================
\newtcolorbox{titlecolorbox}[1]{ %the box around chapter
    coltext=white,
    colframe=primary,
    colback=primary,
    boxrule=0pt,
    arc=0pt,
    notitle,
    width=4.8em,
    height=2.4ex,
    before=\hfill
}


\makeatletter
\let\old@rule\@rule
\def\@rule[#1]#2#3{\textcolor{primary}{\old@rule[#1]{#2}{#3}}}
\makeatother

\titleformat{\chapter}[display]
{\Huge}
{}
{0pt}
{\begin{titlecolorbox}{}
        {\large\MakeUppercase{\bf\chaptername}}
    \end{titlecolorbox}
    \vspace*{-4.19ex}\noindent\rule{\textwidth}{0.4pt}
    \parbox[b]{\dimexpr\textwidth-4.8em\relax}{\raggedright\MakeUppercase{#1}}{\hfill\fontsize{70}{60}\selectfont{\color{primary}\thechapter}}
}
[]

\titleformat{name=\chapter,numberless}[display]
{\Huge\normalfont}
{}
{0pt}
{
    \vspace*{-4.19ex}\noindent\rule{\textwidth}{0.4pt}
    \parbox[b]{\dimexpr\textwidth-4.8em\relax}{\raggedright\MakeUppercase{#1}}
}
[]

%==============
% Secciones
%==============

\titleformat{\section}[hang]{\Large\normalfont\bfseries}
{\rlap{\color{primary}\rule[-6pt]{\textwidth}{0.4pt}}\colorbox{primary}{%
        \raisebox{0pt}[13pt][3pt]{ \makebox[60pt]{% height, width
                \fontfamily{cmr}\selectfont\color{white}{\thesection}}
        }}}%
{15pt}%
{ \color{primary}#1
    %
}
\titlespacing*{\section}{0pt}{3mm}{5mm}

%================
% Sub secciones
%================
\subsectionfont{\Large\color{primary}}

%---------------------
% Portada
%---------------------
\usetikzlibrary{ shapes.geometric }
\usetikzlibrary{calc}
\newcommand{\portada}[3]{
    \begin{tikzpicture}[remember picture,overlay]
        %%%%%%%%%%%%%%%%%%%% Background %%%%%%%%%%%%%%%%%%%%%%%%
        %\fill[primary] (current page.south west) rectangle (current page.north east);


        \foreach \i in {2.5,...,22}
            {
                \node[rounded corners,black!40,draw,regular polygon,regular polygon sides=6, minimum size=\i cm,ultra thick] at ($(current page.west)+(2.5,-5)$) {} ;
            }

        %%%%%%%%%%%%%%%%%%%% Background Polygon %%%%%%%%%%%%%%%%%%%% 
        \foreach \i in {0.5,...,22}
            {
                \node[rounded corners,black!40,draw,regular polygon,regular polygon sides=6, minimum size=\i cm,ultra thick] at ($(current page.north west)+(2.5,0)$) {} ;
            }

        \foreach \i in {0.5,...,22}
            {
                \node[rounded corners,black!20,draw,regular polygon,regular polygon sides=6, minimum size=\i cm,ultra thick] at ($(current page.north east)+(0,-9.5)$) {} ;
            }


        \foreach \i in {21,...,6}
            {
                \node[black!85,rounded corners,draw,regular polygon,regular polygon sides=6, minimum size=\i cm,ultra thick] at ($(current page.south east)+(-0.2,-0.45)$) {} ;
            }


        %%%%%%%%%%%%%%%%%%%% Title of the Report %%%%%%%%%%%%%%%%%%%% 
        \node[left,black!70,minimum width=0.625*\paperwidth,minimum height=3cm, rounded corners] at ($(current page.north east)+(0,-9.5)$)
        {
            {\fontsize{25}{30} \selectfont \bfseries #1} 
        };

        %%%%%%%%%%%%%%%%%%%% Subtitle %%%%%%%%%%%%%%%%%%%% 
        \node[left,black!60,minimum width=0.625*\paperwidth,minimum height=2cm, rounded corners] at ($(current page.north east)+(0,-11)$)
        {
            {\huge \textit{#2}}
        };

        %%%%%%%%%%%%%%%%%%%% Author Name %%%%%%%%%%%%%%%%%%%% 
        \node[left,black!60,minimum width=0.625*\paperwidth,minimum height=2cm, rounded corners] at ($(current page.north east)+(0,-13)$)
        {
            {\Large \textsc{#3}}
        };

        %%%%%%%%%%%%%%%%%%%% Year %%%%%%%%%%%%%%%%%%%% 
        \node[rounded corners,fill=black!70,text =primary!5,regular polygon,regular polygon sides=6, minimum size=2.5 cm,inner sep=0,ultra thick] at ($(current page.west)+(2.5,-5)$) {\LARGE \bfseries \the\year{}};

    \end{tikzpicture}
}

% declaracions de dreceres
\providecommand{\ol}{\overline}
\providecommand{\ul}{\underline}
\providecommand{\wt}{\widetilde}
\providecommand{\wh}{\widehat}
\providecommand{\eps}{\varepsilon}
\providecommand{\half}{\frac{1}{2}}
\providecommand{\inv}{^{-1}}
\newcommand{\dang}{\measuredangle} %% Directed angle
\providecommand{\CC}{\mathbb C}
\providecommand{\s}{\sigma}
\providecommand{\FF}{\mathbb F}
\providecommand{\KK}{\mathbb K}
\providecommand{\NN}{{\mathbb Z_{\geq1}}}
\providecommand{\QQ}{\mathbb Q}
\providecommand{\RR}{\mathbb R}
\providecommand{\ZZ}{\mathbb Z}
\providecommand{\PP}{\mathbb P}
\providecommand{\bigO}{\mathcal O}
\newcommand\smallO{
  \mathchoice
    {{\scriptstyle\mathcal{O}}}% \displaystyle
    {{\scriptstyle\mathcal{O}}}% \textstyle
    {{\scriptscriptstyle\mathcal{O}}}% \scriptstyle
    {\scalebox{.6}{$\scriptscriptstyle\mathcal{O}$}}%\scriptscriptstyle
  }
\providecommand{\dd}{\mathrm{d}}

\providecommand{\ts}{\textsuperscript}
\providecommand{\dg}{^\circ}
\providecommand{\ii}{\item}
\DeclareMathOperator*{\lcm}{lcm}
\DeclareMathOperator*{\argmin}{arg min}
\DeclareMathOperator*{\argmax}{arg max}
\DeclareMathOperator{\Ima}{Im}
\DeclareMathOperator{\Hom}{Hom}
\newcommand{\mcal}[1]{\mathcal{#1}}

\begin{document}
\renewcommand{\rmdefault}{cmr}

\pagestyle{empty}
\portada{Resum de teoria pel final de MATN}{Curs 2025-2026}{Bernat Esteve}


\newpage

\tableofcontents

\newpage
\chapter{Definicions}
\section{Sèries de Dirichlet}
\dfe{Sèrie de Dirichlet}{Una sèrie de Dirichlet és una sèrie de la forma
\[
\sum_{n\in\NN}\frac{f(n)}{n^s}
\]
On $f\colon\NN\rightarrow\CC$ és una funció aritmètica, i $s\in\CC$.
}

\dfe{Convergència de productoris}{
Sigui $\{z_n\}_n\subset\CC$ una successió de complexos, aleshores el productori $\prod z_n$ convergeix si i només si existeix el límit $\lim_{n\rightarrow\infty}\prod_{i=1}^nz_n$, i aquest és no nul.
}

\dfe{Convergència absoluta de productoris}{
Donada una seqüència $z_n$ amb $\Re(z_n)>0$, aleshores el productori $\prod z_r$ es diu que convergeix absolutament si $\sum \log(z_r)$ convergeix absolutament.
}
\section{Funcions L}
\dfe{Caràcter d'un grup}{
Sigui $G$ un grup finit i abelià, aleshores un caràcter de $G$ serà un morfisme $\psi\colon G\rightarrow \CC^*$ (on $\CC^*$ és el grup multiplicatiu de $\CC\setminus\{0\}$)}

\dfe{El grup de caràcters}{Denotem per $\widehat{G}$ al grup $\widehat{G}=\Hom(G,\CC^*)=\{\text{Caràcters de G}\}$, on la operació és:
\[
\psi,\phi\colon G\rightarrow\CC^*\qquad\text{aleshores }(\psi{\;\cdot_{\widehat{G}} } \;\phi)(g)\mapsto\psi(g)\cdot_{\small{\mathbb{C}}^*}\phi(g)
\]
I anomenarem a $\widehat{G}$ el grup de caràcters de $G$.
}

\dfe{Caràcter mòdul $m$}{
Un caràcter mòdul $m$ és un caràcter de $(\mathbb{Z}/m\mathbb{Z})^\times$.}

\dfe{Caràcter principal mòdul $m$}{El caràcter principal mòdul $m$ és $\chi_0(a) \colon \ZZ^+ \to \CC$ definit per 
\[
    \chi_0(a) = \begin{cases}
        1 & \gcd(a, m) = 1 \\
        0 & \text{altrament}
    \end{cases}
\]}

\dfe{Funció $L$ associada a un caràcter de Dirichlet}{La funció $L$ associada al caràcter de Dirichlet mòdul $m$ $\chi$ és la sèrie de Dirichlet \[
L(\chi, s) = \sum_{n\geq 1} \frac{\chi(n)}{n^s}.
\].}

\dfe{Funció $m$-èsima de Dirichlet}{Sigui $m\in\NN$, aleshores la funció $m$-èsima de Dirichlet es defineix com
\[
\zeta_m(s)\vcentcolon=\prod_\chi L(\chi,s)
\]
On el producte recorre tots els caràcters de Dirichlet mòdul $m$.
}

\dfe{Caràcters reals i complexos.}{Diem que $\chi\colon\NN\to \CC$ és un caràcter de Dirichlet real si $\Ima(\chi)\subset\RR$. És a dir $\Ima(\chi)\subset\{-1,0,1\}$. I direm que és complex altrament.}

\dfe{Caràcter conjugat}{Direm $\overline{X}$ al caràcter de Dirichlet conjugat
\[
\overline{X}\colon\NN\to\CC\qquad\text{que envia } a\in\ZZ\text{ a }\overline{\chi}(a)=\overline{\chi(a)}
\]
I no és massa difícil de veure que aquesta funció és un caràcter de Dirichlet del mateix mòdul.}
\section{Continuació meromorfa i equació funcional de $\zeta(s)$}

\dfe{$\theta$ de Jacobi}{
La funció $\theta$ de Jacobi es defineix $\theta\colon \RR_{>0}\to \RR$
\[
\theta(x)=\sum_{n\in\ZZ}e^{-n^2\pi x}
\]
I definim també la funció $\omega$ com
\[
\omega(x)=\sum_{n\geq1}e^{-n^2\pi x}=\frac{\theta(x)-1}{2}
\]
}

\dfe{La funció $\Gamma$}{La funció $\Gamma$ es defineix:
\[
\Gamma(s)=\int_0^\infty e^{-x}x^s\frac{\dd x}{x}\qquad \text{per }\sigma>0
\]}
\dfe{Funció de Riemann completada}{Definim la funció de Riemann completada com
\[
\xi(s)\vcentcolon=\frac{1}{2}s(s-1)\pi^{-\frac{s}{2}}\Gamma\left(\frac{s}{2}\right)\zeta(s)
\]
I també tenim
\[
\frac{s(s-1)}{2}\int_1^\infty \left(x^{\frac{s}{2}-1}+x^{-\frac{s+1}{2}}\right)\omega(x)\dd x+\frac{1}{2}
\]}
\section{Productes d'Hadamard}
\dfe{Ordre d'una funció}{
Sigui $f(s)$ una funció entera. Es diu que $f$ és d'ordre menor o igual que $\alpha\in\RR_{\geq0}$ si existeix $r_0\in\RR_{\geq0}$ tal que
\[
f(s)=\bigO\left(e^{|s|^{\alpha}}\right)\qquad\text{per tot }|s|\geq r_0
\]
Aleshores l'ordre de $f$ es defineix
\[
\inf\{\alpha\in\RR_{\geq0}|f\text{ és d'ordre }\leq \alpha\}.
\]
}
\newpage
\chapter{Enunciar Abel}
\teorema{Criteri de sumació d'Abel}{\label{2.04_teorema_sumació_Abel}Sigui $a\colon\NN\rightarrow\CC$, considerem $A(t)\coloneq\sum_{n\leq t}a(n)$ les sumes parcials de $a$; i una funció $g:\RR_{\geq0}\rightarrow\CC$ amb derivada contínua en un interval $[x,y]\neq\varnothing$\footnote{Per alguna raó, al professor li ha agradat considerar l'interval $[y,x]$, però em nego a fer servir aquesta notació.}. Aleshores 
\[
\sum_{x<n\leq y}a(n)g(n)=A(y)g(y)-A(x)g(x)-\int_x^yA(t)g'(t)\dd t
\]
}

\newpage
\chapter{Demostrar 3.8 (sense veure que la integral I(s) és una funció holomorfa).}
\teorema{Teorema 8 (No donat)}{Per $\sigma>1$, tenim:
\[
\pi^{-\frac{s}{2}}\Gamma\left(\frac{s}{2}\right)\zeta(s)=-\frac{1}{s}+\frac{1}{s-1}+\underbrace{\int_1^\infty\left(x^{\frac{s}{2}-1}+x^{-\frac{s}{2}-\frac{1}{2}}\right)\omega(x)\dd x}_{=I(s)}
\]
On la funció $I(s)$ és una funció holomorfa a tot $\CC$.
}
% \pf{
\begin{enumerate}
    \ii Comencem amb $\Gamma(s)$, i apliquem el canvi de variable $x=n^2\pi x$ i ho treiem tot a fora.\\
    \ii Suposem que hi ha convergència absoluta, i fem el sumatori sobre $n\geq1$ dels 2 costats (en un costat apareix $\omega$ i a l'altre apareix $\zeta$).\\
    \ii Separem la integral $\int_0^\infty=\int_0^1+\int_{1}^{\infty}$.\\
    \ii Fem el canvi de varible $u=\frac{1}{t}$ a la integral de la esquerra.\\
    \ii Fem servir $\theta(\frac{1}{x})=\sqrt{x}\theta(x)$ i que $2\omega(x)+1=\theta$.\\
\end{enumerate}
% }
\newpage
\chapter{Demostració de 4.6 i assumint 4.7.}
\Prop{proposició 6 (Donat)}{Sigui $f$ una funció entera d'ordre $\alpha$. I suposem que $f$ no té zeros. Aleshores es té que $f=e^{g(s)}$ on $g$ és un polinomi de grau $\alpha$. És a dir, que el grau d'una funció entera sense zeros ha de ser un enter.
}
    \Prop{proposició 7 (No donat)}{Sigui $U\subseteq \CC$ simplement connex. Sigui $f:U\to\CC$ holomorfa i sense zeros en $U$. Aleshores existeix $g:U\to\CC$ holomorfa tal que $f(s)=e^{g(s)}$ per tot $s\in U$.}
    Amb això, només ens fa falta veure que $g$ ha de ser un polinomi.
    \begin{enumerate}
        \item Agafar la $g(s)$ que ens dona la proposició.
        \item Fitar la part real de $g(s)$
        \item Menjar-nos el que no fa falta en una constant.
        \item Assumir que $g(0)=0$
        \item Fer Taylor de $g$.
        \item posar els summands de Taylor com a integrals amb el teorema de Cauchy
        \item Calcular la integral de $g(s)s^{n-1}$, i fer el conjugat.
        \item Sumar 0 a $a_n$
        \item Fer el valor absolut de tot $a_n$. 
        \item Integrar $g(s)/s$, i veure que la integral de la part real ha de ser 0.
        \item Sumar 0 a la integral que tenim de $|a_n|$.
    \end{enumerate}
\newpage
\chapter{Demostració de 4.15 i assumint 4.10 i 4.14.}
\teorema{Teorema 15 (Donat)}{
    La funció zeta de Riemann completada
    \[
        \xi(s)=\half s(s-1)\pi^{-s/2}\zeta(s)\Gamma\left(\frac s2\right)
    \]
    satisfà:
    \begin{enumerate}
        \item Es una funció d'ordre 1.
        \item Té un nombre infinit de zeros (a la franja crítica).
        \item $\sum_{n\geq1}\frac{1}{|\rho_n|}$ divergeix; i $\sum_{n\geq1}\frac{1}{|\rho_n|^{1+\eps}}$ convergeix $\forall \eps>0$.
    \end{enumerate}
}
Notem que els zeros de la funció $\xi$ són exactament els zeros (amb multipliciat) de $\zeta$ en la franja crítica.
\begin{enumerate}
    \item Notar que $\half s(s-1)\pi^{-s/2}$ té ordre 1.
    \item Si ens centrem en $\s>0$ la funció gamma també té ordre 1 (i argumentar per què és suficient $\s>0$).
    \item Escriure $\zeta$ com $\frac{1}{s-1}+1-s\int_1^\infty\{x\}x^{-s-1}\dd x$ en $\s>\half$, i veure que això també és d'ordre 1 (de fet és aproximadament lineal per $s\gg1$). 
    % Aquí veiem que $\xi$ té ordre menor o igual que 1, ara ens falta veure que realment té ordre 1.
    \item Argumentar que $\log(r!)\sim r \log(r)$.(una fita és trivial, l'altre es fa comparant-ho amb una integral)
    \item Estudiar $\log(\xi)$ en $s=r$ (un enter)
    \item Estudiar $r\to\infty$.
    \item Com que té ordre 1, pel corol·lari 14 sabem que si el sumatori convergeix aleshores la funció ha de ser $\leq e^{C|s|}$.
    \item Raonar a partir d'aquí que hi ha d'haver un numero infinit de zeros.
    \item Enunciar el corol·lari 10, i acabar la demostració.
\end{enumerate}
\newpage
\chapter{Demostració de 5.5. La demostració ha d'incloure els enunciats i demostracions de 5.2 i), 5.3, i 5.4 i).}
\lema{Lema 2 (No donat)}{\label{6.02_lema}Per $\s>1$ es té:
\begin{enumerate}
    \ii $\Re\;\log(\zeta(s))=\sum_p\sum_{m\geq1}\frac{\cos(mt\log p)}{mp^{\s m}}$\\
\end{enumerate}
On en el segon apartat, la funció $\Lambda(n)=\left\{\begin{array}{cc}
    \log p&\text{ si }n=p^k\\
    0&\text{ altrament.}
\end{array}\right.$
}
\pf{
    \begin{enumerate}
        \item Per corol·lari 1.17 sabem que $\log(\zeta(s))=-\sum_{p}\log(1-p^{-s})=\sum_p\sum_{m\geq 1}\frac{1}{mp^{s m}}$ a tot $\CC$ (per continaució analítica)
        \item trencar $s$ en la seva part real i la seva part imaginària. 
        \item Fer $\Re(\log(\zeta(s)))$.
    \end{enumerate}
}
\lema{Lema de Mertens (no donat)}{\label{6.03_lema de Mertens} Per qualsevol $\theta\in\RR$, tenim la següent desigualtat.
\[
    3+4\cos(\theta)+\cos(2\theta)\geq0
\]
}
\pf{
    \begin{enumerate}
        \item Utilitzar la fórmula de l'angle doble del cosinus.
    \end{enumerate}
}
\Prop{Mertens per a la funció $\zeta$. (No donat)}{\label{6.04_prop}
    Per $\s>1$, tenim:
    \begin{enumerate}
        \ii $\zeta^3(\s)|\zeta^4(\s+it)||\zeta(\s+i2t)|\geq1$,\\
    \end{enumerate}
}
\pf{
    \begin{enumerate}
        \item Aplicar el lema 2.
        \item Aplicar el lema de Mertens.
    \end{enumerate}
}
\teorema{Regió lliure de zeros I (no donat)}{\label{6.05_teorema regió lliure de zeros en sigma=0,1}La funció $\zeta(s)\neq0$ en $\s=1$ (i per la simetria de $\zeta$, en $\s=0$).}
\pf{
\begin{enumerate}
    \item Suposem que $\exists t\in\RR$ tal que $\zeta(1+it)=0$.
    \item Veure que $t\neq 0$
    \item A més $\zeta(\s)\sim \frac{1}{\s-1}$.
    \item agafem $f$ tal que $\zeta(\s+it)=(\s-1)^mf(\s+it)$
    \item Utilitzem la prop 6, i arribem a una contradicció
\end{enumerate}

}
\newpage
\chapter{Demostració del Teorema 5.8 (i el lema 5.6.1 i 5.7).}
\lema{Lema 6 (No donat)}{
    Per $1<\s\leq2$
    \[
        -\frac{\zeta'(s)}{\zeta(s)}<\frac{1}{\s-1}+K\qquad\text{on }K\in\RR
    \]
}
\lema{Lema 7 (No donat)}{
    Sigui $\rho=\beta+i\gamma$ un zero no trivial de $\zeta(s)$ amb $\gamma\geq2$. Sigui $s=\s+i t$ amb $1< \s\leq 2$ i $t\geq2$, aleshores
    \begin{enumerate}
        \item $-\Re\left(\frac{\zeta'(\s+2it)}{\zeta(\s+2it)}\right)<K\log(t)$ per $k\in\RR_{>0}$.
        \item Si a més $t=\gamma$, aleshores: $-\Re\left(\frac{\zeta'(\s+it)}{\zeta(\s+it)}\right)<k\log(t)-\frac{1}{\s-\beta}$ per $k\in\RR_{\geq0}$
    \end{enumerate}
}
\lema{Teorema 8 (Donat)}{
    Existeix $c\in\RR_{>0}$ tal que $\zeta$ no té zeros a la regió $t\geq2,$ i $\s\leq1-\frac{c}{\log(t)}$
}
\pf{
    \begin{enumerate}
        \item Posem el que ens diu Mertens per $-\frac{\zeta'}{\zeta}$.
        \item Posem $s(t)=\s+it$ amb $\s=1+\frac{\delta}{\log(t)}$ de tal manera que estigui entre $\s\in(1,2]$.
        \item Prenem $t$ l'ordenada d'algun zero no trivial de $\zeta$.
        \item Fitem tots els termes pel que ens diuen els lemes 6 i 7.
        \item Fem que totes les $k$ siguin iguals i simplifiquem les coses de tal manera que només quedin 3 termes.
        \item Fem $\s=1+\delta/\log(t)$.
        \item Fem àlgebra i posem tot el que sobra en una constant (que per $\delta$ prou petita, aquesta constant serà positiva).
    \end{enumerate}}
\newpage
\chapter{Demostració de 6.8 assumint 6.9 (l'enunciat del qual cal recordar).}
\Prop{Proposició 8 (No donat)}{Per $T\gg1$, i $x\in\RR_{\geq2}$, i $c=1+\frac{1}{\log x}$, aleshores:
\[
    \frac{1}{2\pi i}\int_{c-iT}^{c+iT}\underbrace{\left(-\frac{\zeta'(s)}{\zeta(s)}\right)\frac{x^s}{s}}_{F(s)}\dd s=x-\sum_{\substack{\rho=\beta+i\gamma\\|\gamma|<T}}\frac{x^{\rho}}{\rho}-\frac{\zeta'(0)}{\zeta(0)}-\frac{1}{2}\log\left(1-\frac{1}{x^2}\right)+\bigO\left(\frac{x(\log T)^2}{T\log(x)}\right)
\]
On la suma sobre $\rho$ és una suma sobre els zeros no trivials de $\zeta$.}
\pf{
\begin{enumerate}
    \item Prenem el contorn $c+iT\overset{\gamma_1}\rightsquigarrow-U+iT\overset{\gamma_2}{\rightsquigarrow}-U-iT\overset{\gamma_3}\rightsquigarrow c-iT\rightsquigarrow c+iT$. Amb $U$ senar positiu, i T no és la ordenadad de cap zero.
    \item Considerem la integral de $F$ sobre aquest domini, i anomenem les integrals $I_1$, $I_2$, $I_3$ a les integrals associades al seu camí respectiu; i $R$ al residu.
    \item Calculem el residu dels 3 tipus de pols de $F$.
    \item Enunciem el lema 9.
    \item Calculem el límit $U\to \infty$.
\end{enumerate}
\lema{Lema 9 (No donat)}{Les integrals:
\begin{align*}    
    I_1,I_3&=\bigO\left(\frac{x(\log T)^2}{T\log x}\right)\\
    I_2&=\bigO\left(\frac{T\log U}{U x^U}\right)
\end{align*}
}
}
\newpage
\chapter{Demostració de 6.5}
\lema{Lema 5 (Donat)}{Sigui 
\[
    \delta(y)=\left\{\begin{array}{cl}0&\text{si }y\in(0,1)\\\frac12&\text{si }y=1\\1&\text{si }y>1\end{array}\right.
\]
Aleshores, per $y>0$, $c>0$, $T>0$:
\[
    \left|\frac{1}{2\pi i}\int_{c-iT}^{c+iT}\frac{y^s}{s}\dd s-\delta(y)\right|<\left\{\begin{array}{cl}
        \frac{y^c}{T|\log y|}&\text{si }y\neq 1\\
        \frac{c}{T}&\text{si }y=1.
    \end{array}\right.
\]
}
\pf{
    \begin{enumerate}
        \item Cas 1: $y\in(0,1)$. Considerem el domini $c-iT\overset{\gamma 1}\rightsquigarrow U-iT\overset{\gamma_2}\rightsquigarrow U+iT\overset{\gamma_3}\rightsquigarrow c+iT$
        \item Fitem la $|I_2|$.
        \item Fem $\lim_{U\to\infty}$.
        \item Fem $|\s-iT|>|T|$, integrem, i ho ajuntem tot.

        \item Cas 2: $y>1$. Considerem el domini $c-iT\overset{\gamma 1}\rightsquigarrow -U-iT\overset{\gamma_2}\rightsquigarrow -U+iT\overset{\gamma_3}\rightsquigarrow c+iT$
        \item Calculem el residu en $s=0$. I calculem les integrals com ho hem fet abans.
        
        \item Cas 3: $y=1$. Calculem la integral directament.
        \item Argumentem per paritat per carregar-nos la part senar. I per la part parella fem un canvir d'integral per tenir la integral de l'arctangent.
        \item Calculem $|I-\delta(y)|$, on fitem $1/1+u^2<1/u^2$ i aconseguim la fita de l'enunciat.
    \end{enumerate}
}

\end{document}